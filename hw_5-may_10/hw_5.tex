\documentclass{article} % A4 paper and 11pt font size
\setcounter{secnumdepth}{0}

\usepackage{amssymb, amsmath, amsfonts}
\usepackage{moreverb}
\usepackage{graphicx}
\usepackage{extarrows}
\usepackage{enumerate}
\usepackage{graphics}
\usepackage{mathrsfs}
\usepackage[margin=1.25in]{geometry}
\usepackage{color}
\usepackage{tocloft}
\renewcommand{\cftsecleader}{\cftdotfill{\cftdotsep}}
\usepackage{array}
\usepackage{float}
\usepackage{hyperref}
\usepackage{textcomp}
\usepackage[makeroom]{cancel}
\usepackage{bbold}
\usepackage{alltt}
\usepackage{physics}
\usepackage{mathtools}
\usepackage[normalem]{ulem}
\usepackage{amsthm}
\usepackage{tikz}
\usetikzlibrary{positioning}
\usetikzlibrary{arrows}
\usepackage{pgfplots}
\usepackage{bigints}
\allowdisplaybreaks
\pgfplotsset{compat=1.12}

\theoremstyle{plain}
\newtheorem*{theorem*}{Theorem}
\newtheorem{theorem}{Theorem}
\newtheorem*{lemma*}{Lemma}
\newtheorem{lemma}{Lemma}

\newenvironment{definition}[1][Definition]{\begin{trivlist}
\item[\hskip \labelsep {\bfseries #1}]}{\end{trivlist}}

\newcommand{\E}{\varepsilon}
\def\Rl{\mathbb{R}}
\def\Cx{\mathbb{C}}

\usepackage[T1]{fontenc} % Use 8-bit encoding that has 256 glyphs
\usepackage{fourier} % Use the Adobe Utopia font for the document - comment this line to return to the LaTeX default
\usepackage[english]{babel} % English language/hyphenation

\usepackage{sectsty} % Allows customizing section commands
\allsectionsfont{\centering \normalfont\scshape} % Make all sections centered, the default font and small caps

\usepackage{fancyhdr} % Custom headers and footers
\pagestyle{fancy} % Makes all pages in the document conform to the custom headers and footers
\fancyhead[L]{\bf Sam Fleischer}
\fancyhead[C]{\bf UC Davis \\ Analysis (MAT201C)} % No page header - if you want one, create it in the same way as the footers below
\fancyhead[R]{\bf Spring 2016}

\fancyfoot[L]{\bf } % Empty left footer
\fancyfoot[C]{\bf \thepage} % Empty center footer
\fancyfoot[R]{\bf } % Page numbering for right footer
\renewcommand{\headrulewidth}{0pt} % Remove header underlines
\renewcommand{\footrulewidth}{0pt} % Remove footer underlines
\setlength{\headheight}{25pt} % Customize the height of the header

\newcommand{\VEC}[2]{\left\langle #1, #2 \right\rangle}
\newcommand{\ran}{\text{\rm ran }}
\newcommand{\spt}{\text{\rm spt }}
\newcommand{\sgn}{\text{\rm sgn }}
\newcommand{\Hilb}{\mathcal{H}}

\DeclareMathOperator*{\esssup}{\text{ess~sup}}

\newcommand{\problem}[2]{
\vspace{.375cm}
\boxed{\begin{minipage}{\textwidth}
    \section{\bf #1}
    #2
\end{minipage}}
}

\numberwithin{equation}{section} % Number equations within sections (i.e. 1.1, 1.2, 2.1, 2.2 instead of 1, 2, 3, 4)
\numberwithin{figure}{section} % Number figures within sections (i.e. 1.1, 1.2, 2.1, 2.2 instead of 1, 2, 3, 4)
\numberwithin{table}{section} % Number tables within sections (i.e. 1.1, 1.2, 2.1, 2.2 instead of 1, 2, 3, 4)

\setlength\parindent{0pt} % Removes all indentation from paragraphs - comment this line for an assignment with lots of text

\newcommand{\horrule}[1]{\rule{\linewidth}{#1}} % Create horizontal rule command with 1 argument of height

\usepackage{xcolor}
\definecolor{light-gray}{gray}{0.9}

\title{ 
\normalfont \normalsize 
\textsc{UC Davis, Analysis (MAT201C), Spring 2016} \\ [25pt] % Your university, school and/or department name(s)
\horrule{2pt} \\[0.4cm] % Thin top horizontal rule
\Huge Homework \#5 \\ % The assignment title
\horrule{2pt} \\[0.5cm] % Thick bottom horizontal rule
}

\author{\huge Sam Fleischer} % Your name

\date{May 10, 2016} % Today's date or a custom date

\begin{document}\thispagestyle{empty}

\maketitle % Print the title

\makeatletter
\@starttoc{toc}
\makeatother

\pagebreak

%%%%%%%%%%%%%%%%%%%%%%%%%%%%%%%%%%%%%%
\problem{Problem 1}{\begin{enumerate}[(a)]
    \item
        For $f \in L^1(\Rl)$, set $S_Rf(x) = \qty(2\pi)^{-\frac{1}{2}}\displaystyle\int_{-R}^R\widehat{f}(\xi)e^{ix\xi}\dd\xi$.  Show that $$S_Rf(x) = K_R * f(x) = \int_{-\infty}^\infty K_R(x-y)f(y)\dd y$$ where $$K_R(x) = \qty(2\pi)^{-1}\int_{-R}^Re^{ix\xi}\dd\xi = \frac{\sin Rx}{\pi x}.$$
    \item
        Show that if $f \in L^2(\Rl)$, then $S_Rf \rightarrow f$ in $L^2(\Rl)$ as $R\rightarrow \infty$.
\end{enumerate}}
\begin{proof}
    \begin{enumerate}[(a)]
        \item
            This proof is simply calculation:
            \begin{align*}
                (K_R * f)(x) &= \int_{-\infty}^\infty K_R(x - y)f(y) \dd y \\
                &= \frac{1}{2\pi}\int_{-\infty}^\infty \int_{-R}^R e^{i(x - y)\xi}\dd\xi f(y) \dd y \\
                &= \frac{1}{2\pi}\int_{-\infty}^\infty \int_{-R}^R e^{ix\xi}e^{-iy\xi} f(y) \dd\xi \dd y \\
                &= \frac{1}{\sqrt{2\pi}}\int_{-R}^R \qty[\frac{1}{\sqrt{2\pi}}\int_{-\infty}^\infty f(y) e^{-iy\xi}\dd y] e^{ix\xi}\dd \xi \\
                &= \frac{1}{\sqrt{2\pi}}\int_{-R}^R\widehat{f}(\xi)e^{ix\xi}\dd\xi \\
                &= S_Rf(x)
            \end{align*}
        \item
            Next up, some analysis!
    \end{enumerate}
\end{proof}






%%%%%%%%%%%%%%%%%%%%%%%%%%%%%%%%%%%%%%
\problem{Problem 2}{Show that for any $R \in (0, \infty)$, there exists $f \in L^1(\Rl)$ such that $S_Rf \not\in L^1(\Rl)$.  Note that $K_R \not\in L^1(\Rl)$.}
\begin{proof}
\end{proof}






%%%%%%%%%%%%%%%%%%%%%%%%%%%%%%%%%%%%%%
\problem{Problem 3}{Assume $w \in \mathcal{S}'(\Rl^n)$ and $w(x) \geq 0$.  Show that if $\widehat{w} \in L^\infty(\Rl^n)$ then $w \in L^1(\Rl^n)$ and $$\norm{\widehat{x}}_{L^\infty(\Rl^n)} = \qty(2\pi)^{-\frac{n}{2}}\norm{x}_{L^1(\Rl^n)}.$$  Hint:  Consider $w_j(x) = \psi\qty(\frac{x}{j})w(x)$ with $\psi \in \mathcal{C}_C^\infty(\Rl^n)$ and $\psi(0) = 1$.  Use the fact that $w_j \rightarrow w$ in $\mathcal{S}'(\Rl^n)$.}
\begin{proof}
\end{proof}






%%%%%%%%%%%%%%%%%%%%%%%%%%%%%%%%%%%%%%
\problem{Problem 4}{Consider the Poisson equation on $\Rl\ :\ u_{xx} = f$.
\begin{enumerate}[(a)]
    \item Show that $\varphi(x) = \frac{x + \abs{x}}{2}$ and $\phi(x) = \frac{\abs{x}}{2}$ are both distributional solutions to $u_{xx} = \delta_0$.
    \item Let $f$ be continuous with compact support in $\Rl$.  Show that $$u(x) = \int_\Rl \varphi(x-y)f(y) \dd y$$ and $$v(x) = \int_\Rl \phi(x - y)f(y) \dd y$$ both solve the Poisson equation $w_{xx}(x) = f(x)$ without relying upon distribution theory.
\end{enumerate}}
\begin{proof}
\end{proof}






%%%%%%%%%%%%%%%%%%%%%%%%%%%%%%%%%%%%%%
\problem{Problem 5}{Let $T \in \mathcal{S}'(\Rl^n)$ and $f \in \mathcal{S}(\Rl^n)$.  Show that the Liebniz rule for distributional derivatives holds: $$\frac{\partial}{\partial x_i}(fT) = f\frac{\partial T}{\partial x_i} + \frac{\partial f}{\partial x_i}T$$ in the sense of distributions.}
\begin{proof}
    \begin{align*}
        \begin{array}{lll}
            \VEC{fT}{\phi'} &= \VEC{T}{f\phi'} & \text{by the definition of multiplication of $\mathcal{S}$ and $\mathcal{S}'$} \\[0.15cm]
            &= \VEC{T}{f\phi'} + \VEC{T}{f'\phi} - \VEC{T}{f'\phi} & \text{by adding and subtracting $\VEC{T}{f'\phi}$} \\[0.15cm]
            &= \VEC{T}{f\phi' + f'\phi} - \VEC{T}{f'\phi} & \text{by linearity of dual pairings} \\[0.15cm]
            &= \VEC{T}{\qty(f\phi)'} - \VEC{T}{f'\phi} \ & \text{by the product rule of functions in $S$} \\[0.15cm]
            &= -\VEC{T'}{f\phi} - \VEC{T}{f'\phi} & \text{by the definition of the distributional derivative of $T \in \mathcal{S}^*$} \\[0.15cm]
            &= -\VEC{T'}{f\phi} - \VEC{f'T}{\phi} & \text{by the definition of multiplication of $\mathcal{S}$ and $\mathcal{S}'$} \\[0.15cm]
            &= -\VEC{fT'}{\phi} - \VEC{f'T}{\phi} & \text{by the definition of multiplication of $\mathcal{S}$ and $\mathcal{S}'$} \\[0.15cm]
            &= -\VEC{fT' + f'T}{\phi} & \text{by linearity of dual pairings}
        \end{array}
    \end{align*}
    Thus, by the definition of distributional derivative, $\qty(fT)' = fT' + f'T$.
\end{proof}






%%%%%%%%%%%%%%%%%%%%%%%%%%%%%%%%%%%%%%
\problem{Problem 6}{Show that a function $f \in L^2(\Rl^n)$ is real if and only if $$\widehat{f}(-\xi) = \overline{\widehat{f}(\xi)}.$$}
\begin{proof}
    First note the following equality:
    \begin{align*}
        \overline{\widehat{f}(\xi)} = \overline{\int_\Rl f(x)e^{-ix\xi}\dd x} = \int_\Rl \overline{f(x)}e^{ix\xi}\dd x = \int_\Rl \overline{f(x)}e^{-ix(-\xi)}\dd x = \widehat{\overline{f}}(-\xi)
    \end{align*}
    If $f$ is real, then $\overline{f} = f$, and thus $\widehat{f}(-\xi) = \overline{\widehat{f}(\xi)}$.  On the other hand, if $\widehat{f}(-\xi) = \overline{\widehat{f}(\xi)}$, then $\widehat{f}(-\xi) = \widehat{\overline{f}}(-\xi)$.  Since $f \in L^2$, then the Fourier transform is a bijection, and thus $f = \overline{f}$, which proves $f$ is a real-valued function.
\end{proof}






\end{document}
