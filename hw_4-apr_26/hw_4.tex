\documentclass{article} % A4 paper and 11pt font size
\setcounter{secnumdepth}{0}

\usepackage{amssymb, amsmath, amsfonts}
\usepackage{moreverb}
\usepackage{graphicx}
\usepackage{extarrows}
\usepackage{enumerate}
\usepackage{graphics}
\usepackage[margin=1.25in]{geometry}
\usepackage{color}
\usepackage{tocloft}
\renewcommand{\cftsecleader}{\cftdotfill{\cftdotsep}}
\usepackage{array}
\usepackage{float}
\usepackage{hyperref}
\usepackage{textcomp}
\usepackage[makeroom]{cancel}
\usepackage{bbold}
\usepackage{alltt}
\usepackage{physics}
\usepackage{mathtools}
\usepackage[normalem]{ulem}
\usepackage{amsthm}
\usepackage{tikz}
\usetikzlibrary{positioning}
\usetikzlibrary{arrows}
\usepackage{pgfplots}
\usepackage{bigints}
\allowdisplaybreaks
\pgfplotsset{compat=1.12}

\theoremstyle{plain}
\newtheorem*{theorem*}{Theorem}
\newtheorem{theorem}{Theorem}
\newtheorem*{lemma*}{Lemma}
\newtheorem{lemma}{Lemma}

\newenvironment{definition}[1][Definition]{\begin{trivlist}
\item[\hskip \labelsep {\bfseries #1}]}{\end{trivlist}}

\newcommand{\E}{\varepsilon}
\def\Rl{\mathbb{R}}
\def\Cx{\mathbb{C}}

\usepackage[T1]{fontenc} % Use 8-bit encoding that has 256 glyphs
\usepackage{fourier} % Use the Adobe Utopia font for the document - comment this line to return to the LaTeX default
\usepackage[english]{babel} % English language/hyphenation

\usepackage{sectsty} % Allows customizing section commands
\allsectionsfont{\centering \normalfont\scshape} % Make all sections centered, the default font and small caps

\usepackage{fancyhdr} % Custom headers and footers
\pagestyle{fancy} % Makes all pages in the document conform to the custom headers and footers
\fancyhead[L]{\bf Sam Fleischer}
\fancyhead[C]{\bf UC Davis \\ Analysis (MAT201C)} % No page header - if you want one, create it in the same way as the footers below
\fancyhead[R]{\bf Spring 2016}

\fancyfoot[L]{\bf } % Empty left footer
\fancyfoot[C]{\bf \thepage} % Empty center footer
\fancyfoot[R]{\bf } % Page numbering for right footer
\renewcommand{\headrulewidth}{0pt} % Remove header underlines
\renewcommand{\footrulewidth}{0pt} % Remove footer underlines
\setlength{\headheight}{25pt} % Customize the height of the header

\newcommand{\VEC}[2]{\left\langle #1, #2 \right\rangle}
\newcommand{\ran}{\text{\rm ran }}
\newcommand{\spt}{\text{\rm spt }}
\newcommand{\sgn}{\text{\rm sgn }}
\newcommand{\Hilb}{\mathcal{H}}

\DeclareMathOperator*{\esssup}{\text{ess~sup}}

\newcommand{\problem}[2]{
\vspace{.375cm}
\boxed{\begin{minipage}{\textwidth}
    \section{\bf #1}
    #2
\end{minipage}}
}

\numberwithin{equation}{section} % Number equations within sections (i.e. 1.1, 1.2, 2.1, 2.2 instead of 1, 2, 3, 4)
\numberwithin{figure}{section} % Number figures within sections (i.e. 1.1, 1.2, 2.1, 2.2 instead of 1, 2, 3, 4)
\numberwithin{table}{section} % Number tables within sections (i.e. 1.1, 1.2, 2.1, 2.2 instead of 1, 2, 3, 4)

\setlength\parindent{0pt} % Removes all indentation from paragraphs - comment this line for an assignment with lots of text

\newcommand{\horrule}[1]{\rule{\linewidth}{#1}} % Create horizontal rule command with 1 argument of height

\usepackage{xcolor}
\definecolor{light-gray}{gray}{0.9}

\title{ 
\normalfont \normalsize 
\textsc{UC Davis, Analysis (MAT201C), Spring 2016} \\ [25pt] % Your university, school and/or department name(s)
\horrule{2pt} \\[0.4cm] % Thin top horizontal rule
\Huge Homework \#3 \\ % The assignment title
\horrule{2pt} \\[0.5cm] % Thick bottom horizontal rule
}

\author{\huge Sam Fleischer} % Your name

\date{April 19, 2016} % Today's date or a custom date

\begin{document}\thispagestyle{empty}

\maketitle % Print the title

\makeatletter
\@starttoc{toc}
\makeatother

\pagebreak

%%%%%%%%%%%%%%%%%%%%%%%%%%%%%%%%%%%%%%
\problem{Problem 1}{Let $f \in L^1(\Rl)$, and set
\begin{align*}
    g(x) = \int_{-\infty}^x f(y) \dd y.
\end{align*}
Prove that $g$ is continuous, and show that $\dfrac{\dd g}{\dd x} = f$, where $\dfrac{\dd g}{\dd x}$ denotes the weak derivative.

Hint: given $\phi \in C_C^\infty(\Rl)$, use the definition of $g$ to obtain
\begin{align*}
    \int_\Rl \phi'(x) g(x) \dd x = \int_\Rl \int_{-\infty}^x \phi'(x) f(y) \dd y \dd x.
\end{align*}
Then write this integral as
\begin{align*}
    \lim_{h\rightarrow 0} \frac{1}{h}\int_\Rl \qty[\phi(x + h) - \phi(x)]g(x) \dd x = -\lim_{h\rightarrow 0}\int_\Rl \int_x^{x+h}f(y)\phi(x)\dd y \dd x.
\end{align*}}
\begin{proof}
    First we show $g$ is continuous.  Let $x_n \rightarrow x$.  Then
    \begin{align*}
        \lim_{x_n \rightarrow x}\abs{g(x_n) - g(x)} &= \lim_{x_n \rightarrow x} \abs{\int_{-\infty}^{x_n}f(y) \dd y - \int_{-\infty}^x f(y) \dd y}\\
        &= \lim_{x_n\rightarrow x} \abs{\int_x^{x_n} f(y) \dd y} \\
        &\leq \lim_{x_n \rightarrow x}\int_x^{x_n}\abs{f(y)} \dd y \\
        &= \begin{cases}
            \lim_{x_n \rightarrow x}\norm{f\mathcal{X}_{[x,x_n]}}_1 &,\ \text{if } x_n > x \\
            \lim_{x_n \rightarrow x}\norm{f\mathcal{X}_{[x_n,x]}}_1 &,\ \text{else}
        \end{cases} \\
        &\leq \lim_{x_n \rightarrow x}\norm{f}_1\norm{\mathcal{X}_{[x,x_n]}}_\infty \qquad \text{without loss of generality} \\
        &= \norm{f}_1 \lim_{x_n \rightarrow x} \norm{\mathcal{X}_{x_n,x}}_\infty \\
        &= 0
    \end{align*}
    Thus $g$ is continuous.  Next we show the weak derivative of $g$ is $f$.  Let $\phi$ be any test function.  Then
    \begin{align*}
        \int_\Rl \phi'(x) g(x) \dd x &= \int_\Rl \phi'(x) \int_{-\infty}^x f(y) \dd y \dd y \\
        &= \int_\Rl \int_{-\infty}^x \phi'(x)f(y) \dd y \dd x \\
        &= \int_\Rl \int_y^\infty \phi'(x) f(y) \dd x \dd y \qquad \text{by Fubini's Theorem} \\
        &= \int_\Rl \qty[\int_y^\infty \phi'(x) \dd x] f(y) \dd y \\
        &= \int_\Rl \phi(y) f(y) \dd y \\
        &= \int_\Rl \phi(x) f(x) \dd x
    \end{align*}
    Thus $f$ is the weak derivative of $g$.
\end{proof}






%%%%%%%%%%%%%%%%%%%%%%%%%%%%%%%%%%%%%%
\problem{Problem 2}{Show that $W^{n,1}(\Rl^n) \subset C(\Rl^n) \cap L^\infty(\Rl^n)$.

Hint:  $u(x) = \displaystyle\int_{-\infty}^0 \dots \int_{-\infty}^0 \frac{\partial^n}{\partial x_1 \dots \partial x_n}u(x+y) \dd y_1 \dots \dd y_n$.}
\begin{proof}
    For ease, let $\alpha_1 = (1, 0, 0, \dots, 0)$, $\alpha_2 = (1, 1, 0, 0, \dots, 0)$, $\dots$, $\alpha_n = (1, 1, 1, \dots, 1)$.  By the hint.
    \begin{align*}
        u(x) &= \int_{-\infty}^0 \dots \int_{-\infty}^0 \frac{\partial^n}{\partial x_1 \dots \partial x_n}u(x+y) \dd y_1 \dots \dd y_n \\
        &= \int_{-\infty}^0 \int_{-\infty}^0 D^{\alpha_n}u(x+y) \dd y_1 \dots \dd y_n \\
        &= \int_{-\infty}^{x_n}\dots\int_{-\infty}^{x_1} D^{\alpha_n}u(t) \dd t_1 \dots \dd t_n
    \end{align*}
    by some change of variables.  Thus,
    \begin{align*}
        \sup\abs{u(x)} &= \sup\abs{\int_{-\infty}^{x_n}\dots\int_{-\infty}^{x_1} D^{\alpha_n}u(t) \dd t_1 \dots \dd t_n} \\
        &= \sup \int_{-\infty}^{x_n}\dots\int_{-\infty}^{x_1} \abs{D^{\alpha_n}u(t)} \dd t_1 \dots \dd t_n \\
        &\leq \sup \int_{\Rl^n} \abs{D^{\alpha_n}u(t)} \dd t_1 \dots \dd t_n \\
        &= \norm{D^{\alpha_n}u(t)}_\infty \\
        &< \infty
    \end{align*}
    since $\abs{\alpha} = n$ and hence $D^{\alpha_n}u(t) \in L^1(\Omega)$.  Thus $u$ is bounded, i.e. $u \in L^\infty(\Rl^n)$.  Next we show $u$ is continuous.  For ease, denote $g_i = D^{\alpha_i}f \in L^1$.  Then
    \begin{align*}
        g_{i-1} = \int_{\infty}^{x_i} g_i(t) \dd x_i
    \end{align*}
    is continuous.  However, this is also the integrand of
\end{proof}






%%%%%%%%%%%%%%%%%%%%%%%%%%%%%%%%%%%%%%
\problem{Problem 3}{If $u \in L_\text{loc}^1(\Rl)$ and if $\dfrac{\dd u}{\dd x} = f \in L^1(\Rl)$, then
\begin{align*}
    u(x) = C + \int_{-\infty}^x f(y) \dd y, \qquad a.e.\ x \in \Rl
\end{align*}
for some constant $C$.}
\begin{proof}
    First let $v(x) \coloneqq C + \int_\infty^x f(y) \dd y$.  Then by problem 1, $\frac{\dd v}{\dd x} = f$.  Then for all test functions $\phi$,
    \begin{align*}
        \int_\Rl u(x) \phi'(x) \dd x = -\int_\Rl f(x)\phi(x) \dd x = \int_\Rl v(x) \phi'(x) \dd x
    \end{align*}
    Every test function is the derivative of some other test function, and so we can say that for all test functions $\psi$,
    \begin{align*}
        \int_\Rl u(x) \psi(x) \dd x = \int_\Rl v(x) \phi(x) \dd x
    \end{align*}
    Since this holds for all test functions, it holds in particular for $\psi = \eta_\E$ for any $\E > 0$.  Then
    \begin{align*}
        \int_\Rl u(x) \eta_\E(x-y) \dd x = \int_\Rl v(x) \eta_\E(x-y) \dd x \qquad \iff \qquad u^\E = v^\E
    \end{align*}
    Since $u^\E \rightarrow u$ and $v^\E \rightarrow v$, and $\lim u^\E = \lim v^\E$, then $u = v$, i.e.
    \begin{align*}
        u(x) = C + \int_{-\infty}^x f(y) \dd y, \qquad a.e.\ x \in \Rl
    \end{align*}
\end{proof}






%%%%%%%%%%%%%%%%%%%%%%%%%%%%%%%%%%%%%%
\problem{Problem 4}{Let $\Omega \coloneqq B\qty(0, \frac{1}{2}) \subset \Rl^2$ denote the open ball of radius $\frac{1}{2}$.  For $x = (x_1,x_2) \in \Omega$, let
\begin{align*}
    u(x_1,x_2) = x_1x_2\log(\abs{\log(\abs{x})})\ \ \text{where}\ \ \abs{x} = \sqrt{x_1^2 + x_2^2}.
\end{align*}
\begin{enumerate}[(a)]
    \item Show that $u \in C^1(\bar{\Omega})$.
    \item Show that $\dfrac{\partial^2 u}{\partial x_j^2} \in C(\bar{\Omega})$ for $j = 1, 2$ but $u \not\in C^2(\bar{\Omega})$.
    \item Show that $u \in H^2(\Omega)$.
\end{enumerate}}
\begin{proof}
    \begin{enumerate}[(a)]
        \item First, we calculate the first partial derivatives:
        \begin{align*}
            \frac{\partial u}{\partial x_i} = \frac{x_i^2 x_j}{\abs{x}^2\abs{\log(\abs{x})}} + x_j \log(\abs{\log(\abs{x})})
        \end{align*}
        Note that as $\abs{x} \rightarrow 0$, then by L'Hospital, each of the above terms $\rightarrow 0$.  Thus $u \in C1(\bar{\Omega})$.
        \item Next, we calculate each non-mixed second partial derivative:
        \begin{align*}
            \frac{\partial^2 u}{\partial x_i^2} &= \frac{2x_ix_j\abs{x}^2\abs{\log(\abs{x})}}{\abs{x}^2\abs{\log(\abs{x})}^2} - \frac{x_i^2}{\abs{x}^2\abs{\log(\abs{x})}^2} + \frac{2x_i^3x_j\abs{\log(\abs{x})}}{\abs{x}^2\abs{\log(\abs{x})}^2} + x_ix_j \\
            &= \frac{2x_ix_j}{\abs{\log(\abs{x})}} - \frac{x_i^2}{\abs{x}^2\abs{\log(\abs{x})}^2} + \frac{2x_i^3x_j}{\abs{x}\abs{\log(\abs{x})}} + x_ix_j
        \end{align*}
        Similar to the first partials, each term $\rightarrow 0$ as $\abs{x} \rightarrow 0$.  Thus $\dfrac{\partial^2 u}{\partial x_i^2} \in C(\bar{\Omega})$ for $i = 1, 2$.  However,
        \begin{align*}
            \frac{\partial^2 u}{\partial x_j \partial x_i} &= \cancelto{0}{\frac{x_i^2}{\abs{x}\abs{\log(\abs{x})}}} + \cancelto{0}{\frac{x_i^2x_j^2}{\abs{x}^4\abs{\log(\abs{x})}^2}} + \cancelto{0}{\frac{2x_i^2x_j^2}{\abs{x}^4\abs{\log(\abs{x})}}} + \cancelto{0}{\frac{x_j^2}{\abs{x}\abs{\log(\abs{x})}}} + \cancelto{\infty}{\abs{\log(\abs{\log(\abs{x})})}}
        \end{align*}
        which diverges to $\infty$ as $\abs{x} \rightarrow 0$.  Thus $u \not\in C^2(\bar{\Omega})$.
        \item Although $\dfrac{\partial^2 u}{\partial x_j \partial x_i}$ is not continuous, it is integrable, and thus there is a $v$ such that for all test functions $\phi$,
        \begin{align*}
            \int_{\Omega} u \frac{\partial^2 \phi}{\partial x_1 \partial x_2} \dd x = (-1)^2 \int_\Omega v \phi \dd x + \int_{\partial B_{\frac{1}{2}}(x)}\qty[{\color{red} something} ]\cdot n \dd s
        \end{align*}
    \end{enumerate}
\end{proof}






%%%%%%%%%%%%%%%%%%%%%%%%%%%%%%%%%%%%%%
\problem{Problem 5}{Prove that $C_C^\infty(\Rl^n)$ is dense in $W^{k,p}(\Rl^n)$ for integers $k \geq 0$ and $1 \leq p < \infty$.}
\begin{proof}
    \begin{align*}
        \norm{\eta_\E * u - u}_{W^{k,p}} = \qty(\sum_{\abs{\alpha} \leq k} \norm{D^\alpha(\eta_\E * u) - D^\alpha u}_{L^p})^{\frac{1}{p}} = \qty(\sum_{\abs{\alpha} \leq k}\norm{\eta_\E * D^\alpha u - D^\alpha u}_{L^p})^{\frac{1}{p}} \rightarrow 0
    \end{align*}
    since each $D^\alpha u \in L^p$ and convolutions approximate functions.  Thus $C^\infty(\Rl^n) \cap W^{k,p}(\Rl^n)$ is dense in $W^{k,p}(\Rl^n)$.  Also, let $u \in C^\infty(\Rl^n) \cap W^{k,p}(\Rl^n)$.  Then for $\Omega \subset\subset \Rl^n$, $\eta_\E * \mathcal{X}_\Omega u \in C_C^\infty(\Rl^n)$.  Thus $C_C^\infty$ is dense in $C_C^\infty \cap W^{k,p}(\Rl^n)$.  This shows $C_C^\infty$ is dense in $W^{k,p}(\Rl^n)$.
\end{proof}






%%%%%%%%%%%%%%%%%%%%%%%%%%%%%%%%%%%%%%
\problem{Problem 6}{Let $\eta_\E$ denote the standard mollifier, and for $u \in H^3(\Rl^3)$, set $u^\E = \eta_\E * u$.  Prove that
\begin{align*}
    \norm{u^\E - e}_{L^\infty(\Rl^3)} \leq C\sqrt{\E}\norm{u}_{H^2(\Rl^3)},
\end{align*}
and that
\begin{align*}
    \norm{u^\E - e}_{L^\infty(\Rl^3)} \leq C\E\norm{u}_{H^3(\Rl^3)}.
\end{align*}}
\begin{proof}
\end{proof}






%%%%%%%%%%%%%%%%%%%%%%%%%%%%%%%%%%%%%%
\problem{Problem 7}{Let $D \coloneqq B(0,1) \subset \Rl^2$ denote the unit disc, and let
\begin{align*}
    u(x) = \qty[-\log\abs{x}]^\alpha.
\end{align*}
Prove that the \emph{weak derivative} of $u$ exists for all $\alpha \geq 0$.}
\begin{proof}
\end{proof}







\end{document}
