\documentclass{article} % A4 paper and 11pt font size
\setcounter{secnumdepth}{0}

\usepackage{amssymb, amsmath, amsfonts}
\usepackage{moreverb}
\usepackage{graphicx}
\usepackage{extarrows}
\usepackage{enumerate}
\usepackage{graphics}
\usepackage[margin=1.25in]{geometry}
\usepackage{color}
\usepackage{tocloft}
\renewcommand{\cftsecleader}{\cftdotfill{\cftdotsep}}
\usepackage{array}
\usepackage{float}
\usepackage{hyperref}
\usepackage{textcomp}
\usepackage[makeroom]{cancel}
\usepackage{bbold}
\usepackage{alltt}
\usepackage{physics}
\usepackage{mathtools}
\usepackage[normalem]{ulem}
\usepackage{amsthm}
\usepackage{tikz}
\usetikzlibrary{positioning}
\usetikzlibrary{arrows}
\usepackage{pgfplots}
\usepackage{bigints}
\allowdisplaybreaks
\pgfplotsset{compat=1.12}

\theoremstyle{plain}
\newtheorem*{theorem*}{Theorem}
\newtheorem{theorem}{Theorem}
\newtheorem*{lemma*}{Lemma}
\newtheorem{lemma}{Lemma}

\newenvironment{definition}[1][Definition]{\begin{trivlist}
\item[\hskip \labelsep {\bfseries #1}]}{\end{trivlist}}

\newcommand{\E}{\varepsilon}
\def\Rl{\mathbb{R}}
\def\Cx{\mathbb{C}}

\usepackage[T1]{fontenc} % Use 8-bit encoding that has 256 glyphs
\usepackage{fourier} % Use the Adobe Utopia font for the document - comment this line to return to the LaTeX default
\usepackage[english]{babel} % English language/hyphenation

\usepackage{sectsty} % Allows customizing section commands
\allsectionsfont{\centering \normalfont\scshape} % Make all sections centered, the default font and small caps

\usepackage{fancyhdr} % Custom headers and footers
\pagestyle{fancy} % Makes all pages in the document conform to the custom headers and footers
\fancyhead[L]{\bf Sam Fleischer}
\fancyhead[C]{\bf UC Davis \\ Analysis (MAT201C)} % No page header - if you want one, create it in the same way as the footers below
\fancyhead[R]{\bf Spring 2016}

\fancyfoot[L]{\bf } % Empty left footer
\fancyfoot[C]{\bf \thepage} % Empty center footer
\fancyfoot[R]{\bf } % Page numbering for right footer
\renewcommand{\headrulewidth}{0pt} % Remove header underlines
\renewcommand{\footrulewidth}{0pt} % Remove footer underlines
\setlength{\headheight}{25pt} % Customize the height of the header

\newcommand{\VEC}[2]{\left\langle #1, #2 \right\rangle}
\newcommand{\ran}{\text{\rm ran }}
\newcommand{\spt}{\text{\rm spt }}
\newcommand{\sgn}{\text{\rm sgn }}
\newcommand{\Hilb}{\mathcal{H}}

\DeclareMathOperator*{\esssup}{\text{ess~sup}}

\newcommand{\problem}[2]{
\vspace{.375cm}
\boxed{\begin{minipage}{\textwidth}
    \section{\bf #1}
    #2
\end{minipage}}
}

\numberwithin{equation}{section} % Number equations within sections (i.e. 1.1, 1.2, 2.1, 2.2 instead of 1, 2, 3, 4)
\numberwithin{figure}{section} % Number figures within sections (i.e. 1.1, 1.2, 2.1, 2.2 instead of 1, 2, 3, 4)
\numberwithin{table}{section} % Number tables within sections (i.e. 1.1, 1.2, 2.1, 2.2 instead of 1, 2, 3, 4)

\setlength\parindent{0pt} % Removes all indentation from paragraphs - comment this line for an assignment with lots of text

\newcommand{\horrule}[1]{\rule{\linewidth}{#1}} % Create horizontal rule command with 1 argument of height

\usepackage{xcolor}
\definecolor{light-gray}{gray}{0.9}

\title{ 
\normalfont \normalsize 
\textsc{UC Davis, Analysis (MAT201C), Spring 2016} \\ [25pt] % Your university, school and/or department name(s)
\horrule{2pt} \\[0.4cm] % Thin top horizontal rule
\Huge Homework \#3 \\ % The assignment title
\horrule{2pt} \\[0.5cm] % Thick bottom horizontal rule
}

\author{\huge Sam Fleischer} % Your name

\date{April 19, 2016} % Today's date or a custom date

\begin{document}\thispagestyle{empty}

\maketitle % Print the title

\makeatletter
\@starttoc{toc}
\makeatother

\pagebreak

%%%%%%%%%%%%%%%%%%%%%%%%%%%%%%%%%%%%%%
\problem{Problem 1}{If $u \in L^p(\Rl^n)$ for $1 \leq p < \infty$, and $u^\E = \eta_\E * u$, for $\eta_\E$ the standard mollifier.  Show that $$u^\E \rightarrow u$$ in $L^p(\Rl^n)$ as $\E \rightarrow 0$.}
\begin{proof}
    \begin{align*}
        \norm{\eta_\E * u - u}_p^p &= \int_{\Rl^n} \abs{\eta_\E * u - u}^p \dd x\\
        &= \int_{\Rl^n} \abs{\int_{\Rl^n} \eta_E(y)u(x-y)\dd y - u(x)}^p \dd x \\
        &= \int_{\Rl^n} \abs{\int_{\Rl^n} \eta_\E(y) (u(x-y) - u(x))\dd y}^p \dd x \\
        &= \int_{\Rl^n}\abs{\int \eta_\E(x-y)(u(y) - u(x))}^p \dd x \\
        &\leq {\color{red}\int_{\Rl^n} \frac{C}{\E^n}\abs{\int \abs{u(y) - u(x)}\dd y}^p \dd x}
    \end{align*}
    $\tilde{u} \in C_C(\Rl^n)$ $\implies \tilde{u}$ is uniformly continuous.
    \begin{align*}
        \norm{u - \tilde{u}}_p \leq \frac{\tilde{\E}}{\E} \\
        \implies \norm{\eta_\E * u - u}_p \leq \norm{\eta_\E * u - \eta_\E * \tilde{u}}_p + \norm{\eta_\E * \tilde{u} - \tilde{u}}_p + \norm{\tilde{u}  - u}_p
    \end{align*}
\end{proof}







%%%%%%%%%%%%%%%%%%%%%%%%%%%%%%%%%%%%%%
\problem{Problem 2}{Let $\Omega$ denote an open and smooth subset of $\Rl^n$.  Prove that $\mathcal{C}_c^\infty(\Omega)$ is dense in $L^p(\Omega)$ for $1 \leq p < \infty$.}
\begin{proof}
    $\Omega$ open $\implies$ smooth Urysohn's Lemma:
    {
    \color{red}
        $\Omega$ open $\subset \Rl^n$, and $C_0$, $C_1$ $\subset \Omega$ disjoint nonempty, then $\exists f\ :\ \Omega \rightarrow [0,1]$, smooth, $f(C_0) = \{0\}$, $f(C_1) = \{1\}$.
    }
    Let $\E > 0$.  Pick $A \subset \Omega$.  By inner and outer regularity of Lebesgue measure, there is a compact subset $K$ of $\Omega$ and $\omega \subset \Omega$ such that $K \subset A \subset \omega$ with $\mu(\omega \setminus A) < \E$, $\mu(A \setminus K) < \E$.

    $\Omega \subset \Rl^n$ implies $\Omega$ is locally compact and Hausdorff, which implies $\exists$ precompact $O,U \subset \Omega$ such that $K \subset O \subset \overline{O} \subset U \subset \overline{U} \subset W$.  Apply smooth Urysohn's Lemma to $K = C_1$ and $\overline{U}\setminus O = C_0$.  $f_k\ :\ \Omega \rightarrow [0,1]$, $f(K) = \{1\}$, $f(\Omega\setminus W) = \{0\}$.
    $$
        \int_\Omega \abs{\mathcal{X}_A - f_k}^p \dd\mu = \int_{A \setminus K} \abs{\mathcal{X}_A - f_k}^p\dd\mu + \int_{W\setminus A}\abs{\mathcal{X}_A - f_k}^p \dd\mu \leq M2\E
    $$
    which implies $C_C^\infty(\Omega)$ dense in ISF (Integral Simple Functions) dense in $L^p(\Omega)$. \\

    The integral is split by $\Omega = (\Omega \setminus W) \cup (W\setminus A) \cup (A\setminus K) \cup K$.  But integral over $\Omega \setminus W$ and over $K$ are $0$ for various reasons.. 
\end{proof}






%%%%%%%%%%%%%%%%%%%%%%%%%%%%%%%%%%%%%%
\problem{Problem 3}{Prove that if $u \in L^1_\text{loc}(\Omega)$ satisfies $\int_\Omega u(x) v(x) \dd x = 0$ for all $v \in \mathcal{C}_c^\infty(\Omega)$, then $u = 0$ a.e.~ in $\Omega$.}
\begin{proof}
    {\color{red}Suppose $u \not\equiv 0$.  Then $\exists E \subset \Omega$ with $\mu(E) > 0$ and $u(x) \neq 0$ for all $x \in E$.  Let $K \subset E$ be compact and set $v = \mathcal{X}_K\sgn(u)$.  Then
        \begin{align*}
            \int_\Omega u(x)v(x)\dd x = \int_K \abs{u(x)}\dd x > 0
        \end{align*}
        This is a contradiction.}

        If $f \in L^p_{\text{loc}}$ and $\eta_\E$ is the standard mollifier, then $\eta_\E * f \rightarrow f$ pointwise a.e.
        $$
            \int \eta_\E(x - y)\mu(y)\dd y = 0 \forall \E > 0
        $$
        $\Omega_\E = \left\{x \in \Omega\ :\ d\qty(x, \Omega^C) \geq \E\right\}$.
\end{proof}






%%%%%%%%%%%%%%%%%%%%%%%%%%%%%%%%%%%%%%
\problem{Problem 4}{Let $u \in L^\infty(\Rl^n)$ and let $\eta_\E$ be a standard mollifier.  For $\E > 0$ consider the sequence $\psi_\E \in L^\infty(\Rl^n)$ such that $$\norm{\psi_\E}_\infty \leq 1\ \forall\E > 0\ \text{ and }\ \psi_\E \rightarrow \psi\ \text{a.e. in }\ \Rl^n,$$ define $$v^\E = \eta_\E * (\psi_\E u)\ \text{ and }\ v = \psi u.$$
\begin{enumerate}[(a)]
    \item Prove that $v^\E \overset{*}{\rightharpoonup} v$ in $L^\infty(\Rl^n)$.
    \item Prove that $v^\E \rightarrow v$ in $L^1(B)$ for every ball $B \subset \Rl^n$.
\end{enumerate}}
\begin{proof}
    \begin{enumerate}[(a)]
        \item We want to show $\phi_{v^\E}(f) \rightarrow \phi_v(f)$ for all $f \in L^1(\Rl)$, where $\phi_v$ and $\phi_{v^\E}$ are the continuous linear functionals associated with $v$ and $v^\E$, respectively.
    \end{enumerate}
\end{proof}






%%%%%%%%%%%%%%%%%%%%%%%%%%%%%%%%%%%%%%
\problem{Problem 5}{For $u \in \mathcal{C}^0(\Rl^n; \Rl)$, $\spt(u)$ is the closure of the set $\left\{x \in \Rl^n\ :\ u(x) \neq 0\right\}$, but this definition may not make sense for functions $u \in L^p(\Omega)$.  For example what is the support of $\mathcal{X}_\mathbb{Q}$, the indicator over the rationals? \\

Let $u\ :\ \Rl^n \rightarrow \Rl$, and let $\{\Omega_\alpha\}_{\alpha \in A}$ denote the collection of all open sets on $\Rl^n$ such that for each $\alpha \in A$, $u = 0$ a.e.~on $\Omega_\alpha$.  Define $\Omega=\displaystyle\bigcup_{\alpha\in A} \Omega_\alpha$.  Prove that $u = 0$ a.e.~on $\Omega$. \\

The support of $u$, $\spt(u)$, is $\Omega^C$, the complement of $\Omega$.  Notice that if $v = w$ a.e.~on $\Rl^n$, then $\spt(v) = \spt(w)$; furthermore, if $u \in \mathcal{C}^0(\Rl^n)$, then $\Omega^C = \overline{\left\{x \in \Rl^n\ :\ u(x) \neq 0\right\}}$.  (Hint: Since $A$ is not necessarily countable, it is not clear that $f = 0$ a.e.~on $\Omega$, so find a countable family $U_n$ of open sets in $\Rl^n$ such that every open set on $\Rl^n$ is the union of some of the sets from $\{U_n\}$.)}
\begin{proof}
    Since $\mathcal{X}_\mathbb{Q}$ is nonzero on $\Rl\setminus\mathbb{Q}$, which is a dense subset of $\Rl$, then $\spt\qty(\mathcal{X}_\mathbb{Q}) = \Rl$.  This is nonsence, however, since $\mathcal{X}_\mathbb{Q}$ is equivalent to $0$ in $L^p(\Rl)$.
\end{proof}






%%%%%%%%%%%%%%%%%%%%%%%%%%%%%%%%%%%%%%
\problem{Problem 6}{Prove that if $u \in L^1(\Rl^n)$ and $v \in L^p(\Rl^n)$ for $1 \leq p \leq \infty$, then $$\spt(u * v) \subset \overline{\spt(u) + \spt(v)}.$$}
\begin{proof}
    Suppose $x \not\in \overline{\spt(u) + \spt(v)}$ and define the set $\qty[x - \spt(u)]$ as the shift of the support of $u$ by the vector $x$:
    \begin{align*}
        \qty[x - \spt(u)] = \left\{y\ :\ x - y \in \spt(u)\right\}
    \end{align*}
    Then
    \begin{align*}
        (u*v)(x) = \int_{\Rl^n} u(x - y)v(y) \dd y = \int_{\qty[x - \spt(u)] \cap \spt(v)} u(x - y)v(y)\dd y
    \end{align*}
    If $x_0 \in \spt(v) \cap \qty[x - \spt(u)]$, then $x_0 \in \spt(v)$ and $x - x_0 = 0 \in \spt(u)$.  Then since $x = \qty(x - x_0) + \qty(x_0)$, then $x \in \spt(u) + \spt(v)$, which is a contradiction since $x \not\in \overline{\spt(u) + \spt(v)}$.  Thus $\qty[x - \spt(u)] \cap \spt(v) = \emptyset$, and therefore
    \begin{align*}
        (u*v)(x) = \int_{\qty[x - \spt(u)] \cap \spt(v)} u(x - y)v(y)\dd y = \int_\emptyset u(x-y) v(y) \dd y = 0
    \end{align*}
    and thus $x \not\in \spt(u*v)$.  This shows
    \begin{align*}
        \spt(u*v) \subset \overline{\spt(u) + \spt(v)}.
    \end{align*}
\end{proof}






%%%%%%%%%%%%%%%%%%%%%%%%%%%%%%%%%%%%%%
\problem{Problem 7}{Suppose that $1 < p < \infty$.  If $\tau_y f(x) = f(x - y)$, show that $f$ belongs to $W^{1,p}(\Rl^n)$ if and only if $\tau_y f$ is a Lipschitz function of $y$ with values in $L^p(\Rl^n)$; that is, $$\norm{\tau_y f - \tau_z f}_p \leq C\abs{y - z}.$$ What happens in the case $p = 1$?}
\begin{proof}
\end{proof}






%%%%%%%%%%%%%%%%%%%%%%%%%%%%%%%%%%%%%%
\problem{Problem 8}{If $u \in W^{1,p}(\Rl^n)$ for some $p \in [1, \infty)$ and $\frac{\partial u}{\partial x_j} = 0$, $j = 1, \dots, n$, on a connected open set $\Omega \subset \Rl^n$, show that $u$ is equal a.e.~to a constant on $\Omega$.  (Hint: approximate $u$ using that $\eta_\E * u \rightarrow u$ in $W^{1,p}(\Rl^n)$, where $\eta_\E$ is a sequence of standard mollifiers.  Show that $\frac{\partial}{\partial x_j}(\eta_\E * u) = 0$ on $\Omega_\E \subset\subset \Omega$ where $\Omega_\E \nearrow \Omega$ as $\E \rightarrow 0$.)\\

More generally, if $\frac{\partial u}{\partial x_j} - f_j \in C(\Omega)$, $1 \leq j \leq n$, show that $u$ is equal a.e.~to a funtion in $\mathcal{C}^1(\Omega)$.}
\begin{proof}
\end{proof}







\end{document}
