\documentclass{article} % A4 paper and 11pt font size
\setcounter{secnumdepth}{0}

\usepackage{amssymb, amsmath, amsfonts}
\usepackage{moreverb}
\usepackage{graphicx}
\usepackage{extarrows}
\usepackage{enumerate}
\usepackage{graphics}
\usepackage[margin=1.25in]{geometry}
\usepackage{color}
\usepackage{tocloft}
\renewcommand{\cftsecleader}{\cftdotfill{\cftdotsep}}
\usepackage{array}
\usepackage{float}
\usepackage{hyperref}
\usepackage{textcomp}
\usepackage[makeroom]{cancel}
\usepackage{bbold}
\usepackage{alltt}
\usepackage{physics}
\usepackage{mathtools}
\usepackage[normalem]{ulem}
\usepackage{amsthm}
\usepackage{tikz}
\usetikzlibrary{positioning}
\usetikzlibrary{arrows}
\usepackage{pgfplots}
\usepackage{bigints}
\allowdisplaybreaks
\pgfplotsset{compat=1.12}

\theoremstyle{plain}
\newtheorem*{theorem*}{Theorem}
\newtheorem{theorem}{Theorem}
\newtheorem*{lemma*}{Lemma}
\newtheorem{lemma}{Lemma}

\newenvironment{definition}[1][Definition]{\begin{trivlist}
\item[\hskip \labelsep {\bfseries #1}]}{\end{trivlist}}

\newcommand{\E}{\varepsilon}
\def\Rl{\mathbb{R}}
\def\Cx{\mathbb{C}}

\usepackage[T1]{fontenc} % Use 8-bit encoding that has 256 glyphs
\usepackage{fourier} % Use the Adobe Utopia font for the document - comment this line to return to the LaTeX default
\usepackage[english]{babel} % English language/hyphenation

\usepackage{sectsty} % Allows customizing section commands
\allsectionsfont{\centering \normalfont\scshape} % Make all sections centered, the default font and small caps

\usepackage{fancyhdr} % Custom headers and footers
\pagestyle{fancy} % Makes all pages in the document conform to the custom headers and footers
\fancyhead[L]{\bf Sam Fleischer}
\fancyhead[C]{\bf UC Davis \\ Analysis (MAT201C)} % No page header - if you want one, create it in the same way as the footers below
\fancyhead[R]{\bf Spring 2016}

\fancyfoot[L]{\bf } % Empty left footer
\fancyfoot[C]{\bf \thepage} % Empty center footer
\fancyfoot[R]{\bf } % Page numbering for right footer
\renewcommand{\headrulewidth}{0pt} % Remove header underlines
\renewcommand{\footrulewidth}{0pt} % Remove footer underlines
\setlength{\headheight}{25pt} % Customize the height of the header

\newcommand{\VEC}[2]{\left\langle #1, #2 \right\rangle}
\newcommand{\ran}{\text{\rm ran }}
\newcommand{\spt}{\text{\rm spt }}
\newcommand{\sgn}{\text{\rm sgn }}
\newcommand{\Hilb}{\mathcal{H}}

\DeclareMathOperator*{\esssup}{\text{ess~sup}}

\newcommand{\problem}[2]{
\vspace{.375cm}
\boxed{\begin{minipage}{\textwidth}
    \section{\bf #1}
    #2
\end{minipage}}
}

\numberwithin{equation}{section} % Number equations within sections (i.e. 1.1, 1.2, 2.1, 2.2 instead of 1, 2, 3, 4)
\numberwithin{figure}{section} % Number figures within sections (i.e. 1.1, 1.2, 2.1, 2.2 instead of 1, 2, 3, 4)
\numberwithin{table}{section} % Number tables within sections (i.e. 1.1, 1.2, 2.1, 2.2 instead of 1, 2, 3, 4)

\setlength\parindent{0pt} % Removes all indentation from paragraphs - comment this line for an assignment with lots of text

\newcommand{\horrule}[1]{\rule{\linewidth}{#1}} % Create horizontal rule command with 1 argument of height

\usepackage{xcolor}
\definecolor{light-gray}{gray}{0.9}

\title{ 
\normalfont \normalsize 
\textsc{UC Davis, Analysis (MAT201C), Spring 2016} \\ [25pt] % Your university, school and/or department name(s)
\horrule{2pt} \\[0.4cm] % Thin top horizontal rule
\Huge Homework \#3 \\ % The assignment title
\horrule{2pt} \\[0.5cm] % Thick bottom horizontal rule
}

\author{\huge Sam Fleischer} % Your name

\date{April 19, 2016} % Today's date or a custom date

\begin{document}\thispagestyle{empty}

\maketitle % Print the title

\makeatletter
\@starttoc{toc}
\makeatother

\pagebreak

%%%%%%%%%%%%%%%%%%%%%%%%%%%%%%%%%%%%%%
\problem{Problem 1}{If $u \in L^p(\Rl^n)$ for $1 \leq p < \infty$, and $u^\E = \eta_\E * u$, for $\eta_\E$ the standard mollifier.  Show that $$u^\E \rightarrow u$$ in $L^p(\Rl^n)$ as $\E \rightarrow 0$.}
\begin{proof}
    First we show the following containment:
    \begin{align*}
        \spt(f+g) \subset \spt(f) \cup \spt(g)
    \end{align*}
    Let $x \in \{x \in \Omega\ :\ (f+g)(x) \neq 0\}$.  Then $(f+g)(x) = f(x) + g(x) \neq 0$.  Then either $f(x) = 0$ or $g(x) = 0$, i.e.~$x\in\spt(f)\cup\spt(g)$.  But $\spt(f)\cup\spt(g)$ is closed, which implies
    \begin{align*}
        \spt(f+g) = \overline{\{x \in \Omega\ :\ (f+g)(x) \neq 0\}} \subset \spt(f) \cup \spt(g).
    \end{align*}

    Now let $\E > 0$ and approximate $u$ by $\tilde{u} \in C_C^0(\Rl^n)$ such that $\norm{u - \tilde{u}}_p < \dfrac{\E}{3}$.  Since $\tilde{u}$ is continuous on a compact set, it is uniformly continuous, and thus $\exists \delta_\E > 0$ such that $\abs{x-y} < \delta_\E \implies \abs{\tilde{u}(x) - \tilde{u}(y)} < \mu(K)^{-\frac{1}{p}}\frac{\E}{3}$ where $K$ is defined below.  Define $\delta = \min\left\{\delta_\E, \frac{1}{2}\right\}$.  Define the set $K = \qty(\overline{\spt(\eta_1) + \spt(\tilde{u})})\cup \spt(\tilde{u})$.  Then since $\mu(\spt(\tilde{u})) < \infty$ and $\mu(\spt(\eta_1)) < \infty$, then $\mu(K) < \infty$.  Then
    \begin{align*}
        \norm{\eta_\delta * u - \eta_\delta * \tilde{u}}_p = \norm{\eta_\delta * (u - \tilde{u}}_p \leq \norm{\eta_\delta}_1\norm{u - \tilde{u}}_p = 1 \cdot \frac{\E}{3} = \frac{\E}{3}.
    \end{align*}
    Let $J = \spt(\eta_\delta * \tilde{u} - \tilde{u})$.  Then $J \subset K$ by the first containment shown.  Finally,
    \begin{align*}
        \norm{\eta_\delta * \tilde{u} - \tilde{u}}_p &\leq \qty[\int_J \abs{\int_{B_\delta(x)}\eta_\delta(x-y)\tilde{u}(y)\dd y}^p \dd x - \tilde{u}(x)]^{\frac{1}{p}} \\
        &\leq \qty[\int_J\qty(\int_{B_\delta(x)}\eta_\delta(x - y)\abs{\tilde{u}(u) - \tilde{u}(x)}\dd y)^p \dd x]^{\frac{1}{p}} \\
        &< \mu(K)^{-\frac{1}{p}}\frac{\E}{3}\qty[\int_J \dd x]^{\frac{1}{p}} \\
        &= \mu(K)^{-\frac{1}{p}}\frac{\E}{3}\mu(J)^{\frac{1}{p}} \\
        &\leq \mu(K)^{-\frac{1}{p}}\frac{\E}{3}\mu(K)^{\frac{1}{p}} \\
        &= \frac{\E}{3}
    \end{align*}
    Thus,
    \begin{align*}
        \norm{\eta_\delta * u - u}_p \leq \norm{\eta_\delta * u - \eta_\delta * \tilde{u}}_p + \norm{\eta_\delta * \tilde{u} - \tilde{u}}_p + \norm{\tilde{u} - u}_p < 3\qty(\frac{\E}{3}) = \E,
    \end{align*}
    which shows, since $\E$ is arbitrarily small, that
    \begin{align*}
        \eta_\delta * u \rightarrow u.
    \end{align*}
\end{proof}







%%%%%%%%%%%%%%%%%%%%%%%%%%%%%%%%%%%%%%
\problem{Problem 2}{Let $\Omega$ denote an open and smooth subset of $\Rl^n$.  Prove that $\mathcal{C}_c^\infty(\Omega)$ is dense in $L^p(\Omega)$ for $1 \leq p < \infty$.}
\begin{proof}
    It suffices to show that any convolved function is in $\mathcal{C}_C^\infty$.  Let $u \in L^p$ and choose $\E > 0$.  Then since
    \begin{align*}
        \frac{\partial}{\partial x}\eta_\E(x-y)u(y) = u(y)\frac{\partial}{\partial x}\eta_\E(x - y) \leq u(y) M(y)
    \end{align*}
    since $\eta_\E(x-y)$ is an arbitrarily smooth compactly supported function, and thus its derivatives are arbitrarily smooth compactly supported functions.  Specifically,
    \begin{align*}
        M(y) = \max_{x \in B_\E(y)}\left\{\eta_\E(x - y)\right\}
    \end{align*}
    Next we show $M$ is continuous.  Choose $\tilde{\E} > 0$.  Then choose $\delta$ such that $\abs{x_1 - x_2} < \delta \implies \abs{\eta_\E(x_1) - \eta_\E(x_2)} < \tilde{\E}$.  Then
    \begin{align*}
        M(y + \delta) = \max_{x \in B_\E(y + \delta)}\left\{\eta_\E(x - y - \delta)\right\} \leq \max_{x \in B_\E(y + \delta)} \left\{\eta_\E(x - y) + \tilde{\E}\right\} = M(y) + \tilde{\E}
    \end{align*}
    since
    \begin{align*}
        \abs{\eta_\E(x - y - \delta) - \eta_\E(x - y)} < \tilde{\E}
    \end{align*}
    Similarly, $-\tilde{\E} < M(y + \delta) - M(y)$.  Thus,
    \begin{align*}
        \abs{M(y + \delta) - M(y)} < \tilde{\E}
    \end{align*}
    and thus $M$ is continuous in $M$.  Since $u \in L^p$ and $M$ is continuous, $u(y)M(y)$ is integrable (and it is also a bounding function of $\frac{\partial}{\partial x} \eta_\E(x-y)u(y)$).  Thus Shkoller Lemma 1.39 applies, and
    \begin{align*}
        \frac{\dd}{\dd x}u^\E = \frac{\dd}{\dd x}\int_\Omega \eta_\E(x-y) u(y) \dd y = \int_\Omega \frac{\dd}{\dd x} \eta_\E(x-y)u(y)\dd y = \int_\Omega u(y)\frac{\dd}{\dd x} \eta_\E(x-y)\dd y = u * \frac{\dd}{\dd x}\eta_\E \in \mathcal{C}_C^0(\Omega)
    \end{align*}
    since the convolution of an $L^p$ function with a continuous function is continuous.  This shows $u^\E \in \mathcal{C}_C^1(\Omega)$.  Now suppose $u^\E \in \mathcal{C}_C^k(\Omega)$.  Then
    \begin{align*}
        \frac{\dd^k}{\dd x^k}\int_\Omega \eta_\E(x-y)u(y)\dd y \in \mathcal{C}_C^0(\Omega) \\
        \implies \int_\Omega \frac{\dd^k}{\dd x^k} \eta_\E(x-y) u(y)\dd y = \int_\Omega u(y) \frac{\dd^k}{\dd x^k} \eta_\E(x-y) \dd y
    \end{align*}
    Thus,
    \begin{align*}
        \frac{\dd^{k+1}}{\dd x^{k+1}} \int_\Omega \eta_\E(x-y)u(y)\dd y = \frac{\dd}{\dd x}\int_\Omega u(y)\frac{\dd^k}{\dd x^k} \eta_\E(x-y)\dd y
    \end{align*}
    By similar arguments as above (which apply since all derivatives of $\eta_\E$ are arbitrarily smooth),
    \begin{align*}
        \frac{\dd^{k+1}}{\dd x^{k+1}} \int_\Omega \eta_\E(x-y)u(y)\dd y = \int_\Omega \frac{\dd}{\dd x} u(y) \frac{\dd^k}{\dd x^k} \eta_\E(x-y)\dd y = \int_\Omega u(y)\frac{\dd^{k+1}}{\dd x^{k+1}} \eta_\E(x-y)\dd y = u * \frac{\dd^{k+1}}{\dd x^{k+1}}\eta_\E \in \mathcal{C}_C^0(\Omega)
    \end{align*}
    since $u$ is $L^p$ and all derivatives of $\eta_\E$ are continuous.  Thus, by induction, $u^\E \in \mathcal{C}_C^k(\Omega)$ for all $k = 1, 2, \dots$, i.e.~$u^\E \in \mathcal{C}_C^\infty(\Omega)$.  By problem one, convolutions are dense in $L^p(\Omega)$, and thus $\mathcal{C}_C^\infty$ is dense in $L^p(\Omega)$.
\end{proof}






%%%%%%%%%%%%%%%%%%%%%%%%%%%%%%%%%%%%%%
\problem{Problem 3}{Prove that if $u \in L^1_\text{loc}(\Omega)$ satisfies $\int_\Omega u(x) v(x) \dd x = 0$ for all $v \in \mathcal{C}_c^\infty(\Omega)$, then $u = 0$ a.e.~in $\Omega$.}
\begin{proof}
    Suppose $u$ satisfies the hypothetical conditions, and also that $u \not\equiv 0$.  Then $\exists E \subset \Omega$ with $\mu(E) > 0$ and $u(x) \neq 0$ for all $x \in E$.  Without loss of generality, suppose $u(x) > 0$ for all $x \in E$.  Next let $K \subset L \subset E$, with $K$ compact and $L$ open.  By Urysohn's Lemma for smooth functions, construct the test function $v$ such that $v(x) = 1$ for all $x \in K$ and $v(x) = 0$ for all $x \in \overline{E^C}$ and $v(x) \in [0,1]$ for all $x \in \overline{E_C}\setminus K$.  Then
    \begin{align*}
        \int_\Omega u(x)v(x)\dd x \geq \int_K \abs{u(x)}\dd x > 0
    \end{align*}
    This is a contradiction.  Thus if $u \in L^1_{\text{loc}}(\Omega)$ satisfies $\int_\Omega u(x)v(x) \dd x = 0$ for all test functions $v$, then $u = 0$ a.e.~in $\Omega$.

        % If $f \in L^p_{\text{loc}}$ and $\eta_\E$ is the standard mollifier, then $\eta_\E * f \rightarrow f$ pointwise a.e.
        % $$
        %     \int \eta_\E(x - y)\mu(y)\dd y = 0 \forall \E > 0
        % $$
        % $\Omega_\E = \left\{x \in \Omega\ :\ d\qty(x, \Omega^C) \geq \E\right\}$.
\end{proof}






%%%%%%%%%%%%%%%%%%%%%%%%%%%%%%%%%%%%%%
\problem{Problem 4}{Let $u \in L^\infty(\Rl^n)$ and let $\eta_\E$ be a standard mollifier.  For $\E > 0$ consider the sequence $\psi_\E \in L^\infty(\Rl^n)$ such that $$\norm{\psi_\E}_\infty \leq 1\ \forall\E > 0\ \text{ and }\ \psi_\E \rightarrow \psi\ \text{a.e. in }\ \Rl^n,$$ define $$v^\E = \eta_\E * (\psi_\E u)\ \text{ and }\ v = \psi u.$$
\begin{enumerate}[(a)]
    \item Prove that $v^\E \overset{*}{\rightharpoonup} v$ in $L^\infty(\Rl^n)$.
    \item Prove that $v^\E \rightarrow v$ in $L^1(B)$ for every ball $B \subset \Rl^n$.
\end{enumerate}}
\begin{proof}
    \begin{enumerate}[(a)]
        \item We want to show $\phi_{v^\E}(f) \rightarrow \phi_v(f)$ for all $f \in L^1(\Rl)$, where $\phi_v$ and $\phi_{v^\E}$ are the continuous linear functionals associated with $v$ and $v^\E$, respectively.
    \end{enumerate}
\end{proof}






%%%%%%%%%%%%%%%%%%%%%%%%%%%%%%%%%%%%%%
\problem{Problem 5}{For $u \in \mathcal{C}^0(\Rl^n; \Rl)$, $\spt(u)$ is the closure of the set $\left\{x \in \Rl^n\ :\ u(x) \neq 0\right\}$, but this definition may not make sense for functions $u \in L^p(\Omega)$.  For example what is the support of $\mathcal{X}_\mathbb{Q}$, the indicator over the rationals? \\

Let $u\ :\ \Rl^n \rightarrow \Rl$, and let $\{\Omega_\alpha\}_{\alpha \in A}$ denote the collection of all open sets on $\Rl^n$ such that for each $\alpha \in A$, $u = 0$ a.e.~on $\Omega_\alpha$.  Define $\Omega=\displaystyle\bigcup_{\alpha\in A} \Omega_\alpha$.  Prove that $u = 0$ a.e.~on $\Omega$. \\

The support of $u$, $\spt(u)$, is $\Omega^C$, the complement of $\Omega$.  Notice that if $v = w$ a.e.~on $\Rl^n$, then $\spt(v) = \spt(w)$; furthermore, if $u \in \mathcal{C}^0(\Rl^n)$, then $\Omega^C = \overline{\left\{x \in \Rl^n\ :\ u(x) \neq 0\right\}}$.  (Hint: Since $A$ is not necessarily countable, it is not clear that $f = 0$ a.e.~on $\Omega$, so find a countable family $U_n$ of open sets in $\Rl^n$ such that every open set on $\Rl^n$ is the union of some of the sets from $\{U_n\}$.)}
\begin{proof}
    Define the basis $B$ of the standard topology on $\Rl^n$ by
    \begin{align*}
        B = \{(a_1,b_1) \times \dots \times (a_n,b_n)\ :\ a_i,b_i \in \mathbb{Q}\ \forall 1 \leq i \leq n\}.
    \end{align*}
    $B$ is clearly countable, and is a basis of the standard topology on $\Rl^n$ because $\mathbb{Q}^n$ is dense in $\Rl^n$.  Since $\Omega_\alpha$ is open for each $\alpha \in A$, then $\Omega_\alpha$ can be written as a union of open sets in $B$:
    \begin{align*}
        \Omega_\alpha = \bigcup_{i=1}^\infty B_{\alpha,i}
    \end{align*}
    where $B_{\alpha,i} \in B$.  The union above is countable since $B$ is countable.  Also, $\bigcup_{\alpha\in A} \Omega_\alpha$ can be re-indexed as
    \begin{align*}
        \Omega = \bigcup_{\alpha\in A}\Omega_\alpha = \bigcup_{\alpha\in A}\bigcup_{i=1}^\infty B_{\alpha,i} = \bigcup_{k=1}^\infty B_k
    \end{align*}
    where $B_k \in B$.  This union is countable since each $\bigcup_{i=1}^\infty B_{\alpha,i}$ is countable and all $B_{\alpha,i} \in B$, which is countable.  Thus,
    \begin{align*}
        \mu\qty(\left\{x\in\Omega\ :\ u(x) \neq 0\right\}) = \mu\qty(\left\{x\in\bigcup_{k=1}^\infty B_k\ :\ u(x) \neq 0\right\}) \underbrace{\leq}_\text{countable additivity} \sum_{k=1}^\infty \mu\Big(\big\{x \in \Omega\ :\ u(x) \neq 0\big\}\Big) = 0
    \end{align*}
    In other words, $x = 0$ a.e.~on $\Omega$.
\end{proof}






%%%%%%%%%%%%%%%%%%%%%%%%%%%%%%%%%%%%%%
\problem{Problem 6}{Prove that if $u \in L^1(\Rl^n)$ and $v \in L^p(\Rl^n)$ for $1 \leq p \leq \infty$, then $$\spt(u * v) \subset \overline{\spt(u) + \spt(v)}.$$}
\begin{proof}
    Suppose $x \not\in \overline{\spt(u) + \spt(v)}$ and define the set $\qty[x - \spt(u)]$ as the shift of the support of $u$ by the vector $x$:
    \begin{align*}
        \qty[x - \spt(u)] = \left\{y\ :\ x - y \in \spt(u)\right\}
    \end{align*}
    Then
    \begin{align*}
        (u*v)(x) = \int_{\Rl^n} u(x - y)v(y) \dd y = \int_{\qty[x - \spt(u)] \cap \spt(v)} u(x - y)v(y)\dd y
    \end{align*}
    If $x_0 \in \spt(v) \cap \qty[x - \spt(u)]$, then $x_0 \in \spt(v)$ and $x - x_0 = 0 \in \spt(u)$.  Then since $x = \qty(x - x_0) + \qty(x_0)$, then $x \in \spt(u) + \spt(v)$, which is a contradiction since $x \not\in \overline{\spt(u) + \spt(v)}$.  Thus $\qty[x - \spt(u)] \cap \spt(v) = \emptyset$, and therefore
    \begin{align*}
        (u*v)(x) = \int_{\qty[x - \spt(u)] \cap \spt(v)} u(x - y)v(y)\dd y = \int_\emptyset u(x-y) v(y) \dd y = 0.
    \end{align*}
    Since $\overline{\spt(u) + \spt(v)}$ is closed, its complement is open.  So $\exists \E$ such that $B_\E(x) \subset \overline{\spt(u) + \spt(v)}^C$.  Thus $(u*v)(x) = 0$ for all $x \in B_\E(x)$.  Then
    \begin{align*}
        B_\E(x)\cap\left\{x\in\Omega\ :\ (u*v)(x) \neq 0\right\} = \emptyset.
    \end{align*}
    So there is a neighborhood around $x$ which does not intersect $\left\{x \in \Omega\ :\ (u*v)(x) \neq 0\right\}$.  Thus,
    \begin{align*}
        x \not\in \overline{\left\{x\in\Omega\ :\ (u*v)(x) \neq 0\right\}} = \spt(u*v).
    \end{align*}
    This shows
    \begin{align*}
        \spt(u*v) \subset \overline{\spt(u) + \spt(v)}.
    \end{align*}
\end{proof}






%%%%%%%%%%%%%%%%%%%%%%%%%%%%%%%%%%%%%%
\problem{Problem 7}{Suppose that $1 < p < \infty$.  If $\tau_y f(x) = f(x - y)$, show that $f$ belongs to $W^{1,p}(\Rl^n)$ if and only if $\tau_y f$ is a Lipschitz function of $y$ with values in $L^p(\Rl^n)$; that is, $$\norm{\tau_y f - \tau_z f}_p \leq C\abs{y - z}.$$ What happens in the case $p = 1$?}
\begin{proof}
\end{proof}






%%%%%%%%%%%%%%%%%%%%%%%%%%%%%%%%%%%%%%
\problem{Problem 8}{If $u \in W^{1,p}(\Rl^n)$ for some $p \in [1, \infty)$ and $\frac{\partial u}{\partial x_j} = 0$, $j = 1, \dots, n$, on a connected open set $\Omega \subset \Rl^n$, show that $u$ is equal a.e.~to a constant on $\Omega$.  (Hint: approximate $u$ using that $\eta_\E * u \rightarrow u$ in $W^{1,p}(\Rl^n)$, where $\eta_\E$ is a sequence of standard mollifiers.  Show that $\frac{\partial}{\partial x_j}(\eta_\E * u) = 0$ on $\Omega_\E \subset\subset \Omega$ where $\Omega_\E \nearrow \Omega$ as $\E \rightarrow 0$.)\\

More generally, if $\frac{\partial u}{\partial x_j} - f_j \in C(\Omega)$, $1 \leq j \leq n$, show that $u$ is equal a.e.~to a funtion in $\mathcal{C}^1(\Omega)$.}
\begin{proof}
    We want to show that $\frac{\partial}{\partial x_i}u^\E = 0$ for any $i = 1, 2, \dots, n$.  This would imply $u^\E = C_\E$, for $C_\E$ some constant.  By Theorem 1.40 in Shkoller's Notes, $u^\E \rightarrow u$ pointwise almost everywhere, and thus would imply $C_\E \rightarrow C \equiv u$.
    \begin{align*}
        \frac{\partial}{\partial x_i} (u^\E) &= \frac{\partial}{\partial x_i} \int_{\Omega_\E} \eta_\E(x) u(x-y) \dd x \\
        &= \int_{\Omega_\E} \frac{\partial}{\partial x_i} \qty[\eta_\E(x) u(x-y)]\dd x.
    \end{align*}
    We can interchange the integral and the derivative since the hypotheses of Theorem 1.39 in Shkoller's Notes holds.  In particular,
    \begin{align*}
        \frac{\partial}{\partial x_i} \qty[\eta_\E(x) u(x-y)] &= \qty[\frac{\partial}{\partial x_i}\eta_\E(x)]u(x-y) + \cancelto{0}{\eta_\E(x)\qty[\frac{\partial}{\partial x_i}u(x-y)]}
    \end{align*}
    since all first partial derivatives of $u$ are assumed to be $0$ on $\Omega$.  Thus,
    \begin{align*}
        \frac{\partial}{\partial x_i} \qty[\eta_\E(x) u(x-y)] = \qty[\frac{\partial}{\partial x_i}\eta_\E(x)]u(x-y) \leq M u(x-y)
    \end{align*}
    where $M$ is the maximum of the $i$\textsuperscript{th} derivative of $\eta_\E$, which is attained since $\eta_\E$ is continuous on a compact set.  Since $u \in W^{1,p}(\Rl^n)$, $Mu(x-y)$ is integrable, and thus Theorem 1.39 holds.  Then
    \begin{align*}
        \frac{\partial}{\partial x_i} (u^\E) &= \int_{\Omega_\E}\qty[\frac{\partial}{\partial x_i}\eta_\E(x)]u(x-y)\dd x \\
        &= \int_{\Omega_\E} \qty[\frac{\partial}{\partial x_i} \eta_\E(x-y)]u(y)\dd y \\
        &= -\int_{\Omega_\E} \qty[\frac{\partial}{\partial y_i}\eta_\E(x-y)]u(y)\dd y \qquad \text{by a suitable change of variables} \\
        &= -\int_{\Omega_\E} \eta_\E(x-y)\frac{\partial}{\partial y_i}u(y)\dd y \qquad \text{by the definition of weak derivative of $u \in W^{1,p}$} \\
        &= 0 \qquad \text{by assumption of all first partial derivatives}
    \end{align*}
    Thus, $u^\E = C_\E$ is constant, which shows $u$ is constant. since $u^\E \rightarrow u$ pointwise a.e.
\end{proof}







\end{document}
