\documentclass{article}

\usepackage{mathtools,amsmath,amsfonts,physics}
\usepackage[margin=0.75in]{geometry}
\renewcommand*{\theenumi}{\thesection.\arabic{enumi}}
\renewcommand*{\theenumii}{\theenumi.\arabic{enumii}}
\renewcommand*{\theenumiii}{\theenumii.\arabic{enumiii}}

\newcommand{\E}{\varepsilon}


\title{Functional Analysis Facts}
\author{Sam Fleischer}
\date{\today}

\begin{document}
    \maketitle
    \section{Three Pillars of Analysis}
    \begin{enumerate}
        \item \textbf{Monotone Convergence Theorem} - If a sequence of non-negative functions is increasing, we can pull the limit through an integral. $$\lim_k \int f_k = \int \lim_k f_k.$$
        \item \textbf{Fatou's Lemma} - For a sequence of non-negative functions, the integral of the lim~inf is less than or equal to the lim~inf of the integral. $$\int \liminf_k f_k \leq \liminf_k \int f_k.$$
        \begin{enumerate}
            \item $f_k = k\mathcal{X}_{\qty(0, \frac{1}{k})}$.
            \begin{enumerate}
                \item $\int_0^1 f_k = 1$ for all $k \implies \liminf_k \int_0^1 f_k = \liminf_k 1 = 1$.
                \item But since $f_k \rightarrow 0$ pointwise a.e., $\liminf_k f_k(x) = 0 \implies \int_0^1 \liminf_k f(x) \dd x = 0$.
            \end{enumerate}
        \end{enumerate}
        \item \textbf{Dominated Convergence Theorem} - If a sequence converges pointwise almost everywhere and is dominated, then it converges in norm to its pointwise limit.  $$\lim_k\int f_k = \int \lim_k f_k = \int f.$$  $$\lim_k\norm{f_k - f}_1 = 0.$$
    \end{enumerate}

    \section{Integrals over Product Spaces}
    \begin{enumerate}
        \item \textbf{Fubini's Theorem} - If a function is integrable on a product space, then the integral over the product space is equal to both iterated integrals.
        \begin{enumerate}
            \item Iterated integrals may exist \emph{without} the existence of the integral over the product space.
        \end{enumerate}
        \item \textbf{Semi-converse of Fubini's Theorem} - If an iterated integral exists of the \emph{absolute value} of a function on a prodct space, then the integral of the product space is equal to both iterated integrals.
        \item \textbf{Tonelli's Theorem} - If a function is non-negative and measurable on a product space, then the integral over the product space is equal to both iterated integrals.
    \end{enumerate}

    \section{$L^p$ Spaces}
    \begin{enumerate}
        \item Convexity is a thing.  $$x^\lambda \leq (1 - \lambda) + \lambda x \qquad \forall \lambda \in (0,1).$$ $$a^\lambda b^{1-\lambda} \leq \lambda a + (1 - \lambda)b \qquad \forall \lambda \in (0,1), \qquad \forall a,b \geq 0.$$
        \item \textbf{H\"{o}lder's Inequality} - For conjugate exponents $p$ and $q$, the $1$-norm of a product of $L^p$ and $L^q$ functions is finite, and the $1$-norm of the product is less than or equal to the product of the norms of the original functions.  $$\norm{fg}_1 \leq \norm{f}_p\norm{g}_q$$.
        \item \textbf{Interpolation Inequality} - For $1 \leq r \leq s \leq t \leq \infty$, if $u$ is in $L^r$ and $L^t$, then $u$ is in $L^s$ and the $s$-norm is less than of equal to the product of the $r$- and $t$-norms raised to the appropriate power. $$\norm{u}_s \leq \norm{u}_r^a \norm{u}_t^{1-a}\qquad \text{where } \frac{1}{s} = \frac{a}{r} + \frac{1-a}{t}.$$ $$L^r \cap L^t \subset L^s.$$
        \item \textbf{Minkowski's Inequality} - For functions in $L^p$, the norm of their sum is less than or equal to the sum of their norms. $$\norm{f + g}_p \leq \norm{f}_p + \norm{g}_p.$$
        \item $L^p$ is a normed linear space.
        \item $L^p$ is a Banach Space (a complete (Cauchy sequences converge) normed linear space).  Steps of the proof:
        \begin{enumerate}
            \item Construct the Cauchy sequence.
            \item Construct a monotone sequence from the Cauchy sequence.
            \item Use Mikowski's Inequality and Triangle Inequality to  show the sequence is uniformly bounded.
            \item Show pointwise convergence of Cauchy sequence using Triangle Inequality.
            \item Use dominated Convergence Theorem to show norm convergence of Cauchy sequence.
        \end{enumerate}
        \item \textbf{Pointwise convergence implies a double implication} - If a sequence of functions converge pointwise, then their norms converge if and only if they converge in norm. $$f_k \rightarrow f \text{ pointwise}\ \ \implies\ \ \Bigg[\norm{f_k - f}_p \rightarrow 0\ \ \iff\ \ \norm{f_k}_p \rightarrow \norm{f}_p\Bigg].$$
        \item \textbf{$L^p$ Comparisons} - For $1 \leq r \leq s \leq t \leq \infty$, if a function in $L^s$ can be written as the sum of functions in $L^r$ and $L^t$.  $$L^s \subset L^r + L^t.$$
        \item \textbf{$L^p$ Comparison for Finite Spaces} - For finite measure spaces, a function in $L^q$ is also in $L^p$ for all $q > p$.  $$L^q \subset L^p.$$
        \item \textbf{Approximation of $L^p$ ($p<\infty$) by Simple Functions} - The set of Simple Functions are dense in $L^p$.
        \item \textbf{Approximation of $L^p$ ($p<\infty$) by Continuous Functions} - For bounded measure spaces, the set of continuous functions is dense in $L^p$.
        \item \textbf{Approximation of $L^p_{\text{loc}}$ by Smooth Functions} - For a function $f$ in $L^p_{\text{loc}}$, its mollified functions:
        \begin{enumerate}
            \item are infinitely differentiable,
            \item converge pointwise to $f$,
            \item converge uniformly to $f$ on compact subsets of the space (given $f$ is continuous), and
            \item converge to $f$ in $L^p_{\text{loc}}$.
        \end{enumerate}
    \end{enumerate}

    \section{Convolutions and (in general) Integral Operators}
    \begin{enumerate}
        \item \textbf{Boundedness of Integral Operators} - An integral operator has bounded norm (and is hence continuous) if both of the absolute iterated integrals of its kernel are bounded (say by $C_1$ and $C_2$). $$\norm{K}_{\mathcal{B}(L^p(\mathbb{R}^n))} \leq C_1^\frac{1}{p}C_2^\frac{1}{q}.$$
        \item \textbf{Cauchy-Young Inequality} - If $p$ and $q$ are conjugate exponents, then for all nonnegative $a$ and $b$, $$ab \leq \frac{a^p}{p} + \frac{b^q}{q}.$$
        \begin{enumerate}
            \item \textbf{Cauchy-Young Inequality with $\delta$} - If $p$ and $q$ are conjugate exponents, the for all nonnegative $a$ and $b$, $$ab \leq \delta a^p + C_\delta b^q, \qquad \delta > 0, \qquad C_\delta = \qty(\delta p)^{-\frac{q}{p}}q^{-1}.$$
        \end{enumerate}
        \item \textbf{Simple Version of Young's Inequality} - For $L^1$ function $k$ and $L^p$ function $f$, the $p$-norm of their convolution is less than or equal to the product of their respective norms.  $$\norm{k * f}_p \leq \norm{k}_1 \norm{f}_p.$$
        \item \textbf{(More general) Young's Inequality for Convolution} - For $L^p$ function $k$ and $L^q$ function $f$, the $r$-norm of their convolution is bounded by the product of their respective norms, given $1 + \frac{1}{r} = \frac{1}{p} + \frac{1}{q}$.  $$\norm{k * f}_r \leq \norm{k}_p\norm{f}_q, \qquad 1 + \frac{1}{r} = \frac{1}{p} + \frac{1}{q}.$$
    \end{enumerate}

    \section{The Dual Space and Weak Topology}
    \begin{enumerate}
        \item \textbf{Norm of an Integral Operator is the Norm of its Kernel} - For conjugate exponents $p$ and $q$, integration of an $L^p$ function against an $L^q$ function is a continuous linear functional on $L^p$ and the operator norm is equal to the norm of the $L^q$ function.  $$F_g(f) = \int fg \qquad \text{and} \qquad \norm{F_g}_\text{op} = \norm{g}_q.$$
        \item \textbf{Riesz Representation Theorem ($1 < p < \infty$)} - For conjugate exponents $p$ and $q$, every bounded (continuous) linear functional on $L^p$ can be represented as an integral operator whose kernel is in $L^q$.  $$\phi \in (L^p)^*\ \ \implies\ \ \exists g \in L^q\ \text{ such that }\ \phi(f) = \int f g\ \ \forall f \in L^p.$$
        \item \textbf{Reflexivity of $L^p$ ($1 < p < \infty$)} - The dual space of the dual space of $L^p$ is isomorphic to $L^p$.
        \item \textbf{Radon-Nikodym Theorem} - If $\mu$ and $\nu$ are two finite measures on a measure space where $\nu$ is absolutely continuous with respect to $\mu$, then there exists an $L^1$ function $h$ to change the measure of integration as follows: $$\int F \dd\nu = \int F h \dd\mu \qquad \text{for every positive measurable function } F.$$
        \item \textbf{Converse to H\"{o}lder's Inequality} - For finite measure spaces, if a product of a measurable function and any simple function is $L^1$, and if the supremum of the $L^1$-norm of the product (for simple functions of $L^p$-norm $1$) is finite, then the measurable function is in $L^q$ and its $L^q$-norm is equal to that supremum.  $$M(g) = \sup_{\norm{f}_p = 1} \left\{\abs{\int_\Omega fg \dd\mu}\ :\ f \text{ is a simple function}\right\} < \infty\ \ \implies\ \ g \in L^q(\Omega)\ \ \text{and}\ \ \norm{g}_q = M(g).$$
        \item \textbf{Alaoglu's Lemma} - The closed unit ball in the dual of a Banach space is compact in the weak-$^*$ topology.
        \item \textbf{Weak Compactness for $L^p(\Omega)$ for $1 < p < \infty$} - Every bounded sequence in $L^p$ has a weakly convergent subsequence.
        \item \textbf{Weak-$^*$ compactness for $L^\infty$} - Every bounded sequence in $L^\infty$ has a weak$^*$ convergent subsequence.
        \item \textbf{Convergence implies weak convergence} - Convergent sequences in $L^p$ are weakly convergent.
        \item \textbf{Weak Limits have Bounded Norms} - The $L^p$ norm of a weak limit is bounded by the lim~inf of the $L^p$ norms of its sequence.
        \item \textbf{Weakly convergent Sequences are bounded} - Weakly convergent $L^p$ sequences have bounded $L^p$ norms.
        \item \textbf{Egoroff's Theorem} - For pointwise convergent sequences on finite domains, there exist arbitrarily small (positive measure) subsets such that the sequence converges uniformly on its complement. $$\forall \E < 0,\ \exists E \subset \Omega\ \text{with}\ \abs{E} < \E\ \text{such that}\ f_k \rightarrow f\ \text{uniformly on}\ \Omega\setminus E.$$
        \item \textbf{thing}
    \end{enumerate}
\end{document}