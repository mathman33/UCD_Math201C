\documentclass{article} % A4 paper and 11pt font size
\setcounter{secnumdepth}{0}

\usepackage{amssymb, amsmath, amsfonts}
\usepackage{moreverb}
\usepackage{graphicx}
\usepackage{extarrows}
\usepackage{enumerate}
\usepackage{graphics}
\usepackage[margin=1.25in]{geometry}
\usepackage{color}
\usepackage{tocloft}
\renewcommand{\cftsecleader}{\cftdotfill{\cftdotsep}}
\usepackage{array}
\usepackage{float}
\usepackage{hyperref}
\usepackage{textcomp}
\usepackage[makeroom]{cancel}
\usepackage{bbold}
\usepackage{alltt}
\usepackage{physics}
\usepackage{mathtools}
\usepackage[normalem]{ulem}
\usepackage{amsthm}
\usepackage{tikz}
\usetikzlibrary{positioning}
\usetikzlibrary{arrows}
\usepackage{pgfplots}
\usepackage{bigints}
\allowdisplaybreaks
\pgfplotsset{compat=1.12}

\theoremstyle{plain}
\newtheorem*{theorem*}{Theorem}
\newtheorem{theorem}{Theorem}
\newtheorem*{lemma*}{Lemma}
\newtheorem{lemma}{Lemma}

\newenvironment{definition}[1][Definition]{\begin{trivlist}
\item[\hskip \labelsep {\bfseries #1}]}{\end{trivlist}}

\newcommand{\E}{\varepsilon}
\def\Rl{\mathbb{R}}
\def\Cx{\mathbb{C}}

\usepackage[T1]{fontenc} % Use 8-bit encoding that has 256 glyphs
\usepackage{fourier} % Use the Adobe Utopia font for the document - comment this line to return to the LaTeX default
\usepackage[english]{babel} % English language/hyphenation

\usepackage{sectsty} % Allows customizing section commands
\allsectionsfont{\centering \normalfont\scshape} % Make all sections centered, the default font and small caps

\usepackage{fancyhdr} % Custom headers and footers
\pagestyle{fancy} % Makes all pages in the document conform to the custom headers and footers
\fancyhead[L]{\bf Sam Fleischer}
\fancyhead[C]{\bf UC Davis \\ Analysis (MAT201C)} % No page header - if you want one, create it in the same way as the footers below
\fancyhead[R]{\bf Spring 2016}

\fancyfoot[L]{\bf } % Empty left footer
\fancyfoot[C]{\bf \thepage} % Empty center footer
\fancyfoot[R]{\bf } % Page numbering for right footer
\renewcommand{\headrulewidth}{0pt} % Remove header underlines
\renewcommand{\footrulewidth}{0pt} % Remove footer underlines
\setlength{\headheight}{25pt} % Customize the height of the header

\newcommand{\VEC}[2]{\left\langle #1, #2 \right\rangle}
\newcommand{\ran}{\text{\rm ran }}
\newcommand{\Hilb}{\mathcal{H}}

\DeclareMathOperator*{\esssup}{\text{ess~sup}}

\newcommand{\problem}[2]{
\vspace{.375cm}
\boxed{\begin{minipage}{\textwidth}
    \section{\bf #1}
    #2
\end{minipage}}
}

\numberwithin{equation}{section} % Number equations within sections (i.e. 1.1, 1.2, 2.1, 2.2 instead of 1, 2, 3, 4)
\numberwithin{figure}{section} % Number figures within sections (i.e. 1.1, 1.2, 2.1, 2.2 instead of 1, 2, 3, 4)
\numberwithin{table}{section} % Number tables within sections (i.e. 1.1, 1.2, 2.1, 2.2 instead of 1, 2, 3, 4)

\setlength\parindent{0pt} % Removes all indentation from paragraphs - comment this line for an assignment with lots of text

\newcommand{\horrule}[1]{\rule{\linewidth}{#1}} % Create horizontal rule command with 1 argument of height

\usepackage{xcolor}
\definecolor{light-gray}{gray}{0.9}

\title{ 
\normalfont \normalsize 
\textsc{UC Davis, Analysis (MAT201C), Spring 2016} \\ [25pt] % Your university, school and/or department name(s)
\horrule{2pt} \\[0.4cm] % Thin top horizontal rule
\Huge Homework \#2 \\ % The assignment title
\horrule{2pt} \\[0.5cm] % Thick bottom horizontal rule
}

\author{\huge Sam Fleischer} % Your name

\date{April 12, 2016} % Today's date or a custom date

\begin{document}\thispagestyle{empty}

\maketitle % Print the title

\makeatletter
\@starttoc{toc}
\makeatother

\pagebreak

%%%%%%%%%%%%%%%%%%%%%%%%%%%%%%%%%%%%%%
\problem{Problem 1}{A function $f \in L^p(\Rl^n)$ is said to be \emph{$L_p$-continuous} if $\tau_h f \rightarrow f$ in $L^p(\Rl^n)$ as $h \rightarrow 0$ in $\Rl^n$, where $\tau_h f (x) = f(x-h)$ is the translation of $f$ by $h$.  Prove that, if $1 \leq p < \infty$, every $f \in L^p(\Rl^n)$ is $L^p$-continuous.  Give a counter-example to show that this result is not true when $p = \infty$.  [Hint:  Approximate an $L^p$ function by a $C_c$ function.]}
\begin{proof}
    Define $f \in L^\infty(\Rl)$ as $f(x) = \mathcal{X}_{[0,1]^n}$ where $\mathcal{X}$ is the characteristic function.  Note that $f(1 - \E) = 1$ for all $\E > 0$.  Let $h$ be a small perturbation, i.e.~$0 < \abs{h} \ll 1$, and choose $\E = \frac{h}{2}$.  Then $\forall x \in (0, \E)$, $\tau_hf(x) = 0$ but $f(x) = 1$, and thus $\abs{\tau_hf(x) - f(x)} = 1$.  This shows that $\forall h > 0$, $\exists$ an interval $I_h$ (of positive measure, $\mu(I_n) > 0$) such that $\abs{\tau_hf(x) - f(x)} = 1$ for all $x \in I_h$.  Thus $\tau_hf \not\rightarrow f$ in $L^\infty(\Rl^n)$.
\end{proof}






%%%%%%%%%%%%%%%%%%%%%%%%%%%%%%%%%%%%%%
\problem{Problem 2}{Show that $L^\infty(\Rl)$ is not separable.  [Hint:  There is an uncountable set $\mathcal{F} \subset L^\infty$ such that $\norm{f-g}_\infty \geq 1$ for all $f, g \in \mathcal{F}$ with $f \neq g$.]}
\begin{proof}
\end{proof}






%%%%%%%%%%%%%%%%%%%%%%%%%%%%%%%%%%%%%%
\problem{Problem 3}{Prove Chebyshev's Inequality:  If $f \in L^p$ ($1 \leq p < \infty$), then for any $\alpha > 0$, $$\mu\qty(\left\{x\ :\ \abs{f(x)} > \alpha\right\}) \leq \qty(\frac{\norm{f}_p}{\alpha})^p.$$  [Note that you can find the proof of this simple fact in many texts but you should see if you can figure it out yourself.  Also, note that this inequality holds for all $0 < p < \infty$.]}
\begin{proof}
\end{proof}






%%%%%%%%%%%%%%%%%%%%%%%%%%%%%%%%%%%%%%
\problem{Problem 4}{Assume that $f,g \in L^1(\Rl^n)$.  Prove that the convolution $$(f*g)(x) = \int_{\Rl^n} f(x-y)g(y)\dd y$$ is measurable and in $L^1(\Rl^n)$.}
\begin{proof}
\end{proof}






%%%%%%%%%%%%%%%%%%%%%%%%%%%%%%%%%%%%%%
\problem{Problem 5}{Let $f_n = \sqrt{n}\mathcal{X}_{\qty(0, \frac{1}{n})}$.  Prove that $f_n$ converges weakly to $0$ in $L^2(0,1)$ and $f_n \rightarrow 0$ in $L^1(0,1)$ but $f_n$ does not converge strongly in $L^2(0,1)$.}
\begin{proof}
    \begin{align*}
        \norm{f_n}_2^2 = \int_0^1 n\mathcal{X}_{\qty(0, \frac{1}{n})} \dd x = \int_0^{\frac{1}{n}}n \dd x = 1
    \end{align*}
    Thus $\norm{f_n}_2 = 1$ for all $n$, and thus does not converge strongly to $0$ in $L^2(0,1)$.
    \begin{align*}
        \norm{f_n}_1 = \int_0^1 \sqrt{n}\mathcal{X}_{\qty(0, \frac{1}{n})} \dd x = \int_0^{\frac{1}{n}} {\sqrt{n}} \dd x = \frac{1}{\sqrt{n}}
    \end{align*}
    Thus $\norm{f_n}_1 \rightarrow 0$ as $n \rightarrow 0$, which shows $f_n \rightarrow 0$ strongly in $L^1(0,1)$.  Let $L \in L^2(0,1)^* \cong L^2(0,1)$.  Thus $\exists \ell \in L^2(0,1)$ such that
    \begin{align*}
        L(f) = \int_0^1 \ell(x)f(x) \dd x
    \end{align*}
    for all $f \in L^2$.  Then
    \begin{align*}
        L(f_n) = \int_0^1 \ell(x)\sqrt{n} \mathcal{X}_{\qty(0, \frac{1}{n})} \dd x \leq \qty(\int_0^1 \abs{\ell(x)}^2\dd x)^{\frac{1}{2}}\qty(\int_0^1 n\mathcal{X}_{\qty(0, \frac{1}{n})})^{\frac{1}{2}} = \norm{\ell}_2\int_0^{\frac{1}{n}}n \dd x = \norm{\ell}_2
    \end{align*}
\end{proof}






%%%%%%%%%%%%%%%%%%%%%%%%%%%%%%%%%%%%%%
\problem{Problem 6}{Find a sequence of functions with the property that $f_j$ converges to $0$ in $L^2(\Omega)$ weakly, to $0$ in $L^{\frac{3}{2}}(\Omega)$ strongly, but it does not converge to $0$ strongly in $L^2(\Omega)$.}
\begin{proof}
\end{proof}






%%%%%%%%%%%%%%%%%%%%%%%%%%%%%%%%%%%%%%
\problem{Problem 7}{Let $f_n$ and $g_n$ denote two sequences in $L^p(\Omega)$ with $1 \leq p \leq \infty$ such that $f_n \rightarrow f$ in $L^p(\Omega)$, and $g_n \rightarrow g$ in $L^p(\Omega)$.  Set $h_n = \max\left\{f_n, g_n\right\}$ and prove that $h_n \rightarrow h$ in $L^p(\Omega)$.}
\begin{proof}
\end{proof}






%%%%%%%%%%%%%%%%%%%%%%%%%%%%%%%%%%%%%%
\problem{Problem 8}{Let $f_n$ be a sequence in $L^p(\Omega)$ with $1 \leq p < \infty$, and let $g_n$ be a bounded sequence in $L^\infty(\Omega)$.  Suppose that $f_n \rightarrow f$ in $L^p(\Omega)$ and that $g_n \rightarrow g$ pointwise a.e.  Prove that $f_ng_n \rightarrow fg$ in $L^p(\Omega)$.}
\begin{proof}
\end{proof}






%%%%%%%%%%%%%%%%%%%%%%%%%%%%%%%%%%%%%%
\problem{Problem 9}{Prove that the space of continuous functions with compat support $\mathcal{C}_c^0(\Rl^n)$ is dense in $L^p(\Rl^n)$ for $1 \leq p < \infty$.}
\begin{proof}
\end{proof}









\end{document}
