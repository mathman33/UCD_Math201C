\documentclass{article} % A4 paper and 11pt font size
\setcounter{secnumdepth}{0}

\usepackage{amssymb, amsmath, amsfonts}
\usepackage{moreverb}
\usepackage{graphicx}
\usepackage{extarrows}
\usepackage{enumerate}
\usepackage{graphics}
\usepackage[margin=1.25in]{geometry}
\usepackage{color}
\usepackage{tocloft}
\renewcommand{\cftsecleader}{\cftdotfill{\cftdotsep}}
\usepackage{array}
\usepackage{float}
\usepackage{hyperref}
\usepackage{textcomp}
\usepackage[makeroom]{cancel}
\usepackage{bbold}
\usepackage{alltt}
\usepackage{physics}
\usepackage{mathtools}
\usepackage[normalem]{ulem}
\usepackage{amsthm}
\usepackage{tikz}
\usetikzlibrary{positioning}
\usetikzlibrary{arrows}
\usepackage{pgfplots}
\usepackage{bigints}
\allowdisplaybreaks
\pgfplotsset{compat=1.12}

\theoremstyle{plain}
\newtheorem*{theorem*}{Theorem}
\newtheorem{theorem}{Theorem}
\newtheorem*{lemma*}{Lemma}
\newtheorem{lemma}{Lemma}

\newenvironment{definition}[1][Definition]{\begin{trivlist}
\item[\hskip \labelsep {\bfseries #1}]}{\end{trivlist}}

\newcommand{\E}{\varepsilon}
\def\Rl{\mathbb{R}}
\def\Cx{\mathbb{C}}

\usepackage[T1]{fontenc} % Use 8-bit encoding that has 256 glyphs
\usepackage{fourier} % Use the Adobe Utopia font for the document - comment this line to return to the LaTeX default
\usepackage[english]{babel} % English language/hyphenation

\usepackage{sectsty} % Allows customizing section commands
\allsectionsfont{\centering \normalfont\scshape} % Make all sections centered, the default font and small caps

\usepackage{fancyhdr} % Custom headers and footers
\pagestyle{fancy} % Makes all pages in the document conform to the custom headers and footers
\fancyhead[L]{\bf Sam Fleischer}
\fancyhead[C]{\bf UC Davis \\ Analysis (MAT201C)} % No page header - if you want one, create it in the same way as the footers below
\fancyhead[R]{\bf Spring 2016}

\fancyfoot[L]{\bf } % Empty left footer
\fancyfoot[C]{\bf \thepage} % Empty center footer
\fancyfoot[R]{\bf } % Page numbering for right footer
\renewcommand{\headrulewidth}{0pt} % Remove header underlines
\renewcommand{\footrulewidth}{0pt} % Remove footer underlines
\setlength{\headheight}{25pt} % Customize the height of the header

\newcommand{\VEC}[2]{\left\langle #1, #2 \right\rangle}
\newcommand{\ran}{\text{\rm ran }}
\newcommand{\Hilb}{\mathcal{H}}

\DeclareMathOperator*{\esssup}{\text{ess~sup}}

\newcommand{\problem}[2]{
\vspace{.375cm}
\boxed{\begin{minipage}{\textwidth}
    \section{\bf #1}
    #2
\end{minipage}}
}

\numberwithin{equation}{section} % Number equations within sections (i.e. 1.1, 1.2, 2.1, 2.2 instead of 1, 2, 3, 4)
\numberwithin{figure}{section} % Number figures within sections (i.e. 1.1, 1.2, 2.1, 2.2 instead of 1, 2, 3, 4)
\numberwithin{table}{section} % Number tables within sections (i.e. 1.1, 1.2, 2.1, 2.2 instead of 1, 2, 3, 4)

\setlength\parindent{0pt} % Removes all indentation from paragraphs - comment this line for an assignment with lots of text

\newcommand{\horrule}[1]{\rule{\linewidth}{#1}} % Create horizontal rule command with 1 argument of height

\usepackage{xcolor}
\definecolor{light-gray}{gray}{0.9}

\title{ 
\normalfont \normalsize 
\textsc{UC Davis, Analysis (MAT201C), Spring 2016} \\ [25pt] % Your university, school and/or department name(s)
\horrule{2pt} \\[0.4cm] % Thin top horizontal rule
\Huge Homework \#2 \\ % The assignment title
\horrule{2pt} \\[0.5cm] % Thick bottom horizontal rule
}

\author{\huge Sam Fleischer} % Your name

\date{April 12, 2016} % Today's date or a custom date

\begin{document}\thispagestyle{empty}

\maketitle % Print the title

\makeatletter
\@starttoc{toc}
\makeatother

\pagebreak

%%%%%%%%%%%%%%%%%%%%%%%%%%%%%%%%%%%%%%
\problem{Problem 1}{A function $f \in L^p(\Rl^n)$ is said to be \emph{$L_p$-continuous} if $\tau_h f \rightarrow f$ in $L^p(\Rl^n)$ as $h \rightarrow 0$ in $\Rl^n$, where $\tau_h f (x) = f(x-h)$ is the translation of $f$ by $h$.  Prove that, if $1 \leq p < \infty$, every $f \in L^p(\Rl^n)$ is $L^p$-continuous.  Give a counter-example to show that this result is not true when $p = \infty$.  [Hint:  Approximate an $L^p$ function by a $C_c$ function.]}
\begin{proof}
    We begin with the counter-example for $L^\infty$.  Define $f \in L^\infty(\Rl)$ as $f(x) = \mathcal{X}_{[0,1]^n}$ where $\mathcal{X}$ is the characteristic function.  Note that $f(1 - \E) = 1$ for all $\E > 0$.  Let $h$ be a small perturbation, i.e.~$0 < \abs{h} \ll 1$, and choose $\E = \frac{h}{2}$.  Then $\forall x \in (0, \E)$, $\tau_hf(x) = 0$ but $f(x) = 1$, and thus $\abs{\tau_hf(x) - f(x)} = 1$.  This shows that $\forall h > 0$, $\exists$ an interval $I_h$ (of positive measure, $\mu(I_n) > 0$) such that $\abs{\tau_hf(x) - f(x)} = 1$ for all $x \in I_h$.  Thus $\tau_hf \not\rightarrow f$ in $L^\infty(\Rl^n)$.

    Now let $p \in [1,\infty)$ and let $f \in L^p(\Rl^n)$.  Then by the density of simple functions in $L^p$, there exist simple functions $f_\ell \in L^p$ such that $\norm{f_\ell - f}_p \rightarrow 0$.  By the definition of simple functions, there are disjoint, finite-measure sets $E_{\ell,k}$, where $k = 1, \dots, N_\ell$ and constants $a_{\ell,k}$ such that
    \begin{align*}
        f_\ell = \sum_{k=1}^{N_\ell} a_{\ell,k} \mathcal{X}_{E_{\ell,k}}
    \end{align*}
    Next we show that for each $h > 0$, $\tau_h f_\ell \rightarrow \tau_h f$ in $L^p$.
    \begin{align*}
        \norm{\tau_h f_\ell - \tau_h f}_p^p = \int_{\Rl^n} \abs{f_\ell(x + h) - f(x + h)}^p \dd x = \int_{\Rl^n} \abs{f_\ell(x) - f(x)}^p \dd x = \norm{f_\ell - f}_p^p \rightarrow 0
    \end{align*}
    by a change of variables.  Next we show that for each $n$, $\tau_h f_\ell \rightarrow f_\ell$ as $h \rightarrow 0$.  For each $E_{\ell,k}$, define the set $E_{\ell,k,h}$ as the shifted set $E_{\ell,k}$ by $h$:
    \begin{align*}
        E_{\ell,k,h} = \left\{x + h\ :\ x \in E_{\ell,k}\right\}.
    \end{align*}
    Then
    \begin{align*}
        \norm{\tau_h f_\ell - f_\ell}_p^p &= \int_{\Rl^n} \abs{\sum_{k=1}^{N_\ell}a_{\ell,k}\mathcal{X}_{E_{\ell,k,h}} - \sum_{k=1}^{N_\ell} a_{\ell,k}\mathcal{X}_{E_{\ell,k}}}^p \dd x \\
        &= \int_{\Rl^n} \abs{\sum_{k=1}^{N_\ell} a_{\ell,k}\qty(\mathcal{X}_{E_{\ell,k,h}\setminus E_{\ell,k}} - \mathcal{X}_{E_{\ell,k}\setminus E_{\ell,k,h}})}^p \dd x \\
        &= \int_{\Rl^n} G_{\ell,h} \dd x
    \end{align*}
    where
    \begin{align*}
        G_{\ell,h} = \abs{\sum_{k=1}^{N_\ell} a_{\ell,k}\qty(\mathcal{X}_{E_{\ell,k,h}\setminus E_{\ell,k}} - \mathcal{X}_{E_{\ell,k}\setminus E_{\ell,k,h}})}^p.
    \end{align*}
    Then since $\mu\qty(E_{\ell,k,h}\setminus E_{\ell,k}) \rightarrow 0$ and $\mu\qty(E_{\ell,k}\setminus E_{\ell,k,h}) \rightarrow 0$ as $h \rightarrow 0$, then $G_{\ell,h} \rightarrow 0$ pointwise as $h \rightarrow 0$.  Also,
    \begin{align*}
        \abs{G_{\ell,h}(x)} \leq \abs{\sum_{k=1}^{N_\ell} a_{\ell,k} 2\mathcal{X}_{E_{\ell,k}}(x)}^p = 2^p\abs{\sum_{k=1}^{N_\ell}a_{\ell,k}\mathcal{X}_{E_{\ell,k}}(x)} = 2^p\abs{f_\ell(x)}^p
    \end{align*}\tabularnewline
    That is, for a given $n$, $G_{\ell,k}$ is uniformly bounded by $2^p\abs{f_\ell(x)}^p$.  Then by the Dominated Convergence Theorem, the integral of $G_{\ell,k}$ converges to the integral of its pointwise limit, which is $0$.  Thus, as $h \rightarrow 0$,
    \begin{align*}
        \norm{\tau_h f_\ell - f_\ell}_p^p \rightarrow \int_{\Rl^n} 0 \dd x = 0.
    \end{align*}
    Thus,
    \begin{align*}
        \norm{\tau_h f - f}_p \leq \norm{\tau_h f - \tau_h f_\ell}_p + \norm{\tau_h f_\ell - f_\ell}_p + \norm{f_\ell - f}_p \rightarrow 0
    \end{align*}
    since each of the three norms on the right hand side converge to $0$.
\end{proof}






%%%%%%%%%%%%%%%%%%%%%%%%%%%%%%%%%%%%%%
\problem{Problem 2}{Show that $L^\infty(\Rl)$ is not separable.  [Hint:  There is an uncountable set $\mathcal{F} \subset L^\infty$ such that $\norm{f-g}_\infty \geq 1$ for all $f, g \in \mathcal{F}$ with $f \neq g$.]}
\begin{proof}
    Let $\mathcal{F} = \{\mathcal{X}_{[0,\alpha]}\ :\ 0 < \alpha \in \Rl\}$.  $\mathcal{F}$ is clearly uncountable.  Consider any two $f,g \in \mathcal{F}$.  Then, without loss of generality, $f = \mathcal{X}_{[0,\alpha]}$ and $g = \mathcal{X}_{[0,\beta]}$ where $\alpha < \beta$.  Also, 
    \begin{align*}
        \norm{f - g}_\infty = \esssup\{\mathcal{X}_{(\alpha,\beta]}\} = 1
    \end{align*}
    Thus the ball around any $f \in \mathcal{F}$ of radius $\frac{1}{2}$, i.e.~$B\qty(f,\frac{1}{2})$, contains no other elements of $\mathcal{F}$.  Thus $L^\infty(\Rl)$ is not separable since $\mathcal{F}$ is uncountable and not dense.
\end{proof}






%%%%%%%%%%%%%%%%%%%%%%%%%%%%%%%%%%%%%%
\problem{Problem 3}{Prove Chebyshev's Inequality:  If $f \in L^p$ ($1 \leq p < \infty$), then for any $\alpha > 0$, $$\mu\qty(\left\{x\ :\ \abs{f(x)} > \alpha\right\}) \leq \qty(\frac{\norm{f}_p}{\alpha})^p.$$  [Note that you can find the proof of this simple fact in many texts but you should see if you can figure it out yourself.  Also, note that this inequality holds for all $0 < p < \infty$.]}
\begin{proof}
    Let $A_\alpha = \left\{x\ :\ \abs{f(x)} > \alpha\right\} = \left\{x\ :\ \abs{\frac{f(x)}{\alpha}} > 1\right\} = \left\{x\ :\ \abs{\frac{f(x)}{\alpha}}^p > 1\right\}$ for all $p \geq 1$.  Then
    \begin{align*}
        \qty(\frac{\norm{f}_p}{\alpha})^p = \int_\Omega \abs{\frac{f(x)}{\alpha}}^p \dd \mu = \int_{A_\alpha} \abs{\frac{f(x)}{\alpha}}^p \dd \mu + \int_{\Omega \setminus A_\alpha} \abs{\frac{f(x)}{\alpha}}^p \dd \mu \geq \int_{A_\alpha} \abs{\frac{f(x)}{\alpha}}^p \dd \mu \geq \int_{A_\alpha} 1 \dd \mu = \mu(A_\alpha)
    \end{align*}
    which proves the result.
\end{proof}






%%%%%%%%%%%%%%%%%%%%%%%%%%%%%%%%%%%%%%
\problem{Problem 4}{Assume that $f,g \in L^1(\Rl^n)$.  Prove that the convolution $$(f*g)(x) = \int_{\Rl^n} f(x-y)g(y)\dd y$$ is measurable and in $L^1(\Rl^n)$.}
\begin{proof}
    First note that $\norm{f}_1\norm{g}_1 < \infty$ since they are in $L^1\qty(\Rl^n)$.  Then
    \begin{align*}
        \norm{f}_1\norm{g}_1 &= \norm{f}_1 \int_{\Rl^n}\abs{g(y)} \dd y \\
        &= \int_{\Rl^n} \norm{f}_1 \abs{g(y)} \dd y \\
        &= \int_{\Rl^n} \int_{\Rl^n} \abs{f(x-y)} \dd x \abs{g(y)} \dd y \\
        &= \int_{\Rl^n} \int_{\Rl^n} \abs{f(x-y)} \abs{g(y)}\dd x \dd y \\
        \implies \int_{\Rl^{2n}} \abs{f(x-y)g(y)} \dd x \dd y &< \infty
    \end{align*}
    Thus, by Fubini's Theorem,
    \begin{align*}
        \int_{\Rl^n}\int_{\Rl^n} \abs{f(x-y)g(y)} \dd y \dd x &< \infty \\
        \implies \norm{f * g}_1 = \int_{\Rl^n} \abs{\qty(f * g)(x)} \dd x = \int_{\Rl^n} \abs{\int_{\Rl^n} f(x-y)g(y) \dd y} \dd x \leq \int_{\Rl^n} \int_{\Rl^n} \abs{f(x-y)g(y)} \dd y \dd x &< \infty
        % \implies \int_{\Rl^n} \abs{f(x-y)g(y)} \dd y &< \infty \\
        % \implies \int_{\Rl^n} f(x-y)g(y) \dd y &< \infty \\
        % \implies \qty(f * g)(x) &< \infty
    \end{align*}
    which shows $(f * g) \in L^1(\Rl^n)$.  Since all $L^1$ functions are measurable, $(f * g)$ is measurable.
\end{proof}






%%%%%%%%%%%%%%%%%%%%%%%%%%%%%%%%%%%%%%
\problem{Problem 5}{Let $f_n = \sqrt{n}\mathcal{X}_{\qty(0, \frac{1}{n})}$.  Prove that $f_n$ converges weakly to $0$ in $L^2(0,1)$ and $f_n \rightarrow 0$ in $L^1(0,1)$ but $f_n$ does not converge strongly in $L^2(0,1)$.}
\begin{proof}
    \begin{align*}
        \norm{f_n}_2^2 = \int_0^1 n\mathcal{X}_{\qty(0, \frac{1}{n})} \dd x = \int_0^{\frac{1}{n}}n \dd x = 1
    \end{align*}
    Thus $\norm{f_n}_2 = 1$ for all $n$, and thus does not converge strongly to $0$ in $L^2(0,1)$.
    \begin{align*}
        \norm{f_n}_1 = \int_0^1 \sqrt{n}\mathcal{X}_{\qty(0, \frac{1}{n})} \dd x = \int_0^{\frac{1}{n}} {\sqrt{n}} \dd x = \frac{1}{\sqrt{n}}
    \end{align*}
    Thus $\norm{f_n}_1 \rightarrow 0$ as $n \rightarrow 0$, which shows $f_n \rightarrow 0$ strongly in $L^1(0,1)$.  Let $L \in L^2(0,1)^* \cong L^2(0,1)$.  Thus $\exists \ell \in L^2(0,1)$ such that
    \begin{align*}
        L(f) = \int_0^1 \ell(x)f(x) \dd x
    \end{align*}
    for all $f \in L^2$.  Then
    \begin{align*}
        L(f_n) = \int_0^1 \ell(x)\sqrt{n} \mathcal{X}_{\qty(0, \frac{1}{n})} \dd x = \int_0^{\frac{1}{n}}\ell(x)\sqrt{n}\dd x \leq \qty(\int_0^{\frac{1}{n}}\abs{\ell(x)}^2\dd x)^{\frac{1}{2}}\qty(\int_{0}^{\frac{1}{n}}n\dd x)^{\frac{1}{2}} = \qty(\int_0^{\frac{1}{n}}\abs{\ell(x)}^2\dd x) \rightarrow 0
    \end{align*}
    since $\ell$ is fixed and $\mu\qty(\qty(0, \frac{1}{n})) \rightarrow 0$ as $n \rightarrow 0$.  Thus $f_n \rightharpoonup 0$ in $L^2(0,1)$.
\end{proof}






%%%%%%%%%%%%%%%%%%%%%%%%%%%%%%%%%%%%%%
\problem{Problem 6}{Find a sequence of functions with the property that $f_j$ converges to $0$ in $L^2(\Omega)$ weakly, to $0$ in $L^{\frac{3}{2}}(\Omega)$ strongly, but it does not converge to $0$ strongly in $L^2(\Omega)$.}
\begin{proof}
    Let $f_n = \sqrt{n}\mathcal{X}_{\qty(0, \frac{1}{n})}$.  Then by number 5, $f_n \not\rightarrow 0$ in $L^2(0,1)$ but $f_n \rightharpoonup 0$ in $L^2(0,1)$.  Also,
    \begin{align*}
        \norm{f_n}_{\frac{3}{2}}^{\frac{3}{2}} = \int_0^1\abs{n^{\frac{1}{2}}\mathcal{X}_{\qty(0, \frac{1}{n})}}^{\frac{3}{2}}\dd x = \int_0^\frac{1}{n}n^{\frac{3}{4}}\dd x = n^{-\frac{1}{4}} \rightarrow 0\ \ \text{as}\ \ n \rightarrow \infty
    \end{align*}
    Thus $f_n \rightarrow 0$ in $L^{\frac{3}{2}}(0,1)$.
\end{proof}






%%%%%%%%%%%%%%%%%%%%%%%%%%%%%%%%%%%%%%
\problem{Problem 7}{Let $f_n$ and $g_n$ denote two sequences in $L^p(\Omega)$ with $1 \leq p \leq \infty$ such that $f_n \rightarrow f$ in $L^p(\Omega)$, and $g_n \rightarrow g$ in $L^p(\Omega)$.  Set $h_n = \max\left\{f_n, g_n\right\}$ and prove that $h_n \rightarrow h$ in $L^p(\Omega)$.}
\begin{proof}
    First note that
    \begin{align*}
        h_n(x) &= \frac{1}{2}\qty(f_n(x) + g_n(x)) + \frac{1}{2}\abs{f_n(x) - g_n(x)}, \qquad \text{and} \\
        h(x) &= \frac{1}{2}\qty(f(x) + g(x)) + \frac{1}{2}\abs{f(x) - g(x)}.
    \end{align*}
    Then
    \begin{align*}
        \norm{h_n - h}_p &= \norm{\frac{1}{2}\qty(f_n + g_n) + \frac{1}{2}\abs{f_n - g_n} - \qty[\frac{1}{2}\qty(f + g) + \frac{1}{2}\abs{f - g}]} \\
        &\leq \frac{1}{2}\qty[\norm{f_n - f}_p + \norm{g_n - g}_p + \norm{\abs{f_n - g_n} - \abs{f - g}}_p] \\
        &\leq \frac{1}{2}\qty[\norm{f_n - f}_p + \norm{g_n - g}_p + \norm{(f_n - g_n) - (f - g)}_p] \\
        &\leq \frac{1}{2}\qty[\norm{f_n - f}_p + \norm{g_n - g}_p + \norm{f_n - f}_p + \norm{g_n - g}_p] \\
        &= \norm{f_n - f}_p + \norm{g_n + g}_p
    \end{align*}
    But since $f_n \rightarrow f$ and $g_n \rightarrow g$ in $L^p$, then there is an $N$ such that $\norm{f_n - f}_p < \frac{\E}{2}$ and $\norm{g_n - g}_p < \frac{\E}{2}$ for all $n \geq N$.  Thus if $n \geq N$,
    \begin{align*}
        \norm{h_n - h}_p \leq \norm{f_n - f}_p + \norm{g_n + g}_p < \frac{\E}{2} + \frac{\E}{2} = \E
    \end{align*}
    Since $\E$ is arbitrary, $h_n \rightarrow h$ in $L^p$.
\end{proof}






%%%%%%%%%%%%%%%%%%%%%%%%%%%%%%%%%%%%%%
\problem{Problem 8}{Let $f_n$ be a sequence in $L^p(\Omega)$ with $1 \leq p < \infty$, and let $g_n$ be a bounded sequence in $L^\infty(\Omega)$.  Suppose that $f_n \rightarrow f$ in $L^p(\Omega)$ and that $g_n \rightarrow g$ pointwise a.e.  Prove that $f_ng_n \rightarrow fg$ in $L^p(\Omega)$.}
\begin{proof}
    \begin{align*}
        \norm{f_ng_n - fg}_p &\leq \norm{f_ng_n - fg_n}_p + \norm{fg_n - fg}_p \\
        &= \qty[\int_\Omega \abs{g_n}^p\abs{f_n - f}^p]^{\frac{1}{p}} + \qty[\int_\Omega \abs{fg_n - fg}^p]^{\frac{1}{p}}
    \end{align*}
    Since $g_n$ is a bounded sequence in $L^\infty$, $\exists M$ such that $\abs{g_n(x)} \leq \norm{g_n} \leq M$ almost everywhere.  Thus
    \begin{align*}
        \qty[\int_\Omega \abs{g_n}^p\abs{f_n - f}^p]^{\frac{1}{p}} &\leq \qty[\int_\Omega M^p \abs{f_n - f}^p]^{\frac{1}{p}} = M\norm{f_n - f}_p \rightarrow 0
    \end{align*}
    since $f_n \rightarrow f$ in $L^p$.  Next note $g \in L^\infty$ since
    \begin{align*}
        \norm{g}_\infty = \esssup\{g(x)\} = \esssup\left\{\lim_n g_n(x)\right\} \leq M
    \end{align*}
    Since $g_n \rightarrow g$ pointwise, then $fg_n \rightarrow fg$ pointwise and thus $\abs{fg_n - fg}^p \rightarrow 0$ pointwise.  Define $h_n$ as
    \begin{align*}
        h_n = \abs{fg_n - fg}^p.
    \end{align*}
    Then $h_n \in L^1(\Omega)$ since
    \begin{align*}
        \norm{h_n}_1^p = \norm{fg_n - fg}_p \leq \norm{fg_n}_p + \norm{fg}_p \leq \norm{f}_p\norm{g_n}_\infty + \norm{f}_p\norm{g}_\infty \leq 2\norm{f}_pM
    \end{align*}
    Also, $\abs{h_n(x)}^{\frac{1}{p}} = \abs{(fg_n - fg)(x)} \leq \abs{f(x)g_n(x) - f(x)g(x)} \leq \abs{f(x)}\abs{g_n(x)} + \abs{f(x)}\abs{g(x)} \leq 2M\abs{f(x)}$, which implies $h$ is dominated:
    \begin{align*}
        \abs{h(x)} \leq 2^pM^p\abs{f(x)}^p
    \end{align*}
    Thus, by the dominated convergence theorem,
    \begin{align*}
        \lim_n \int_\Omega h_n = \int_\Omega \lim_n h_n = \int_\Omega 0 = 0 \\
        \implies \lim_n \int_\Omega \abs{fg_n - fg}^p = 0
    \end{align*}
    Thus,
    \begin{align*}
        \lim_n\norm{f_ng_n - fg}_p = \lim_n \qty[\int_\Omega \abs{g_n}^p\abs{f_n - f}^p]^{\frac{1}{p}} + \lim_n \qty[\int_\Omega \abs{fg_n - fg}^p]^{\frac{1}{p}} = 0
    \end{align*}
    which shows $f_ng_n \rightarrow fg$ in $L^p(\Omega)$.
\end{proof}






%%%%%%%%%%%%%%%%%%%%%%%%%%%%%%%%%%%%%%
\problem{Problem 9}{Prove that the space of continuous functions with compat support $\mathcal{C}_c^0(\Rl^n)$ is dense in $L^p(\Rl^n)$ for $1 \leq p < \infty$.}
\begin{proof}
    First we will show that $\mathcal{C}_c^0(\Rl^n)$ is dense in the space of characteristic functions.  Then since simple functions are dense in $L^p(\Rl^n)$ and simple functions are finite linear combinations of simple functions, this will show $\mathcal{C}_c^0(\Rl^n)$ is dense in $L^p(\Rl^n)$.

    Let $\mathcal{X}_A$ be the characteristic function on a bounded, measurable subset $A \subset \Rl^n$.  By the alternate definition of the Lebesgue measure, for any $\E > 0$, there is an open set $G \supset A$ and compact set $K \subset A$ such that $\mu(G\setminus K) < \E$.  By Urysohn's Lemma, there is some continuous function $g$ such that
    \begin{align*}
        g(x) \in \begin{cases}
            \{1\} & \text{if } x \in K \\
            \{0\} & \text{if } x \in G^C \\
            [0,1] & \text{if } x \in G \setminus K
        \end{cases}
    \end{align*}
    Then
    \begin{align*}
        \norm{g - \mathcal{X}_A}_p^p = \int_{G\setminus K} \abs{g(x) - \mathcal{X}_A(x)}^p \dd x \leq \int_{G\setminus K} 1^p \dd x = \mu(G\setminus K) < \E
    \end{align*}
    Thus there are continuous functions that are arbitrarily close in $L^p$-norm to any characteristic function on bounded measurable sets.  Now let $f \in L^p(\Rl^n)$.  Then define $f_n$ as
    \begin{align*}
        f_n(x) = \begin{cases}
            f(x) &\ \text{if } \abs{x} \leq n \\
            0 &\ \text{if } \abs{x} > n
        \end{cases}.
    \end{align*}
    Then by definition, each $f_n$ is compactly supported and converges to $f$ in $L^p$.  Since simple functions are dense in the space of compactly supported functions, so define $g_{n}$ as
    \begin{align*}
        h_n = \sum_{k=1}^{N_n}a_{n,k}\mathcal{X}_{n,k}
    \end{align*}
    where $a_{n,k}$ are constants and $\mathcal{X}_{n,k}$ are bounded measurable sets, and $\norm{f_n - h_n}_p < \E$.  This is possible since for each $\mathcal{X}_{n,k}$, we can define $g_{n,k}$ such that
    \begin{align*}
        \norm{g_{n,k} - a_{n,k}\mathcal{X}_{n,k}} < \frac{\E}{\displaystyle\max_{k=1,\dots,N_n}\{a_{n,k}\}N_n}.
    \end{align*}
    Then define the continuous function $\tilde{g} = \sum_{k=1}^{N_n} g_{n,k}$.  Then
    \begin{align*}
        \norm{f - g}_p &\leq \norm{f - f_n}_p + \norm{f_n - h_n}_p + \norm{h_n - g}_p \\
        &\leq \norm{f - f_n}_p + \norm{f_n - h_n}_p + \sum_{k=1}^{N_n}\norm{a_{n,k}\mathcal{X}_{n,k} - g_{n,k}}_p \rightarrow 0
    \end{align*}
    Thus there is a continuous function $g$ that arbitrarily approximates any $L^p$ function $f$.
\end{proof}









\end{document}
