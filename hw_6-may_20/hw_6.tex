\documentclass{article} % A4 paper and 11pt font size
\setcounter{secnumdepth}{0}

\usepackage{amssymb, amsmath, amsfonts}
\usepackage{moreverb}
\usepackage{graphicx}
\usepackage{extarrows}
\usepackage{enumerate}
\usepackage{graphics}
\usepackage{mathrsfs}
\usepackage[margin=1.25in]{geometry}
\usepackage{color}
\usepackage{tocloft}
\renewcommand{\cftsecleader}{\cftdotfill{\cftdotsep}}
\usepackage{array}
\usepackage{float}
\usepackage{hyperref}
\usepackage{textcomp}
\usepackage[makeroom]{cancel}
\usepackage{bbold}
\usepackage{alltt}
\usepackage{physics}
\usepackage{mathtools}
\usepackage[normalem]{ulem}
\usepackage{amsthm}
\usepackage{tikz}
\usetikzlibrary{positioning}
\usetikzlibrary{arrows}
\usepackage{pgfplots}
\usepackage{bigints}
\allowdisplaybreaks
\pgfplotsset{compat=1.12}

\theoremstyle{plain}
\newtheorem*{theorem*}{Theorem}
\newtheorem{theorem}{Theorem}
\newtheorem*{lemma*}{Lemma}
\newtheorem{lemma}{Lemma}

\newenvironment{definition}[1][Definition]{\begin{trivlist}
\item[\hskip \labelsep {\bfseries #1}]}{\end{trivlist}}

\newcommand{\E}{\varepsilon}
\def\Rl{\mathbb{R}}
\def\Cx{\mathbb{C}}

\usepackage[T1]{fontenc} % Use 8-bit encoding that has 256 glyphs
\usepackage{fourier} % Use the Adobe Utopia font for the document - comment this line to return to the LaTeX default
\usepackage[english]{babel} % English language/hyphenation

\usepackage{sectsty} % Allows customizing section commands
\allsectionsfont{\centering \normalfont\scshape} % Make all sections centered, the default font and small caps

\usepackage{fancyhdr} % Custom headers and footers
\pagestyle{fancy} % Makes all pages in the document conform to the custom headers and footers
\fancyhead[L]{\bf Sam Fleischer}
\fancyhead[C]{\bf UC Davis \\ Analysis (MAT201C)} % No page header - if you want one, create it in the same way as the footers below
\fancyhead[R]{\bf Spring 2016}

\fancyfoot[L]{\bf } % Empty left footer
\fancyfoot[C]{\bf \thepage} % Empty center footer
\fancyfoot[R]{\bf } % Page numbering for right footer
\renewcommand{\headrulewidth}{0pt} % Remove header underlines
\renewcommand{\footrulewidth}{0pt} % Remove footer underlines
\setlength{\headheight}{25pt} % Customize the height of the header

\newcommand{\VEC}[2]{\left\langle #1, #2 \right\rangle}
\newcommand{\ran}{\text{\rm ran }}
\newcommand{\spt}{\text{\rm spt }}
\newcommand{\sgn}{\text{\rm sgn }}
\newcommand{\pval}[1]{\text{\rm p.v.}\qty(#1)}
\newcommand{\Hilb}{\mathcal{H}}

\DeclareMathOperator*{\esssup}{\text{ess~sup}}

\newcommand{\problem}[2]{
\vspace{.375cm}
\boxed{\begin{minipage}{\textwidth}
    \section{\bf #1}
    #2
\end{minipage}}
}

\numberwithin{equation}{section} % Number equations within sections (i.e. 1.1, 1.2, 2.1, 2.2 instead of 1, 2, 3, 4)
\numberwithin{figure}{section} % Number figures within sections (i.e. 1.1, 1.2, 2.1, 2.2 instead of 1, 2, 3, 4)
\numberwithin{table}{section} % Number tables within sections (i.e. 1.1, 1.2, 2.1, 2.2 instead of 1, 2, 3, 4)

\setlength\parindent{0pt} % Removes all indentation from paragraphs - comment this line for an assignment with lots of text

\newcommand{\horrule}[1]{\rule{\linewidth}{#1}} % Create horizontal rule command with 1 argument of height

\usepackage{xcolor}
\definecolor{light-gray}{gray}{0.9}

\title{ 
\normalfont \normalsize 
\textsc{UC Davis, Analysis (MAT201C), Spring 2016} \\ [25pt] % Your university, school and/or department name(s)
\horrule{2pt} \\[0.4cm] % Thin top horizontal rule
\Huge Homework \#6 \\ % The assignment title
\horrule{2pt} \\[0.5cm] % Thick bottom horizontal rule
}

\author{\huge Sam Fleischer} % Your name

\date{May 20, 2016} % Today's date or a custom date

\begin{document}\thispagestyle{empty}

\maketitle % Print the title

\makeatletter
\@starttoc{toc}
\makeatother

\pagebreak

%%%%%%%%%%%%%%%%%%%%%%%%%%%%%%%%%%%%%%
\problem{Problem 1}{Given $f(x) = \frac{1}{\qty(1 + x^2)^2}$ find $\widehat{f}(\xi)$.  Prove that $\widehat{f} \in C^2$.  You can use the following fact that follows from complex integration: $$\int_{-\infty}^\infty \frac{\cos(ax)}{x^2 + b^2}\dd x = \frac{\pi}{b}e^{-ab}, \qquad a,b > 0.$$}
\begin{proof}
    Let $g = \sqrt{f} = \frac{1}{1 + x^2}$.  Then
    \begin{align*}
        \widehat{g} = \int_\Rl \frac{e^{-2\pi i x \xi}}{1 + x^2}\dd x &= \int_\Rl \frac{\cos(2\pi x \xi) - i \sin(2\pi x \xi)}{1 + x^2} \dd x \\
        &= \int_\Rl \frac{\cos(2\pi x \xi)}{1 + x^2}\dd x - \cancelto{0}{i\int_\Rl\frac{\sin(2\pi x \xi)}{1 + x^2}\dd x} \\
        &= \pi e^{-\abs{2\pi\xi}} \\
        \implies \widehat{f} = \widehat{g^2} = \widehat{g}*\widehat{g} = \int_\Rl \pi^2 e^{-\abs{2\pi y} - \abs{2\pi(\xi - y)}}\dd y &= \boxed{\frac{\pi}{2}e^{-\abs{2\pi \xi}}\qty(1 + \abs{2\pi\xi})}
    \end{align*}
    Note that
    \begin{align*}
        \widehat{f}(\xi) &= \frac{\pi}{2}\begin{cases}
            e^{-x}(1 + x) & \text{ if } x \geq 0 \\
            e^{x}(1 - x) & \text{ if } x < 0
        \end{cases} \\
        \implies \widehat{f}'(\xi) &= \frac{\pi}{2}\begin{cases}
            -xe^{-x} & \text{ if } x \geq 0 \\
            -xe^{x} & \text{ if } x > 0
        \end{cases} \\
        \implies \widehat{f}''(\xi) &= \frac{\pi}{2}\begin{cases}
            e^{-x}(x - 1) & \text{ if } x \geq 0 \\
            -e^{x}(x + 1) & \text{ if } x < 0
        \end{cases} \\
        \implies \widehat{f}'''(\xi) &= \frac{\pi}{2}\begin{cases}
            -e^{-x}(x - 2) & \text{ if } x \geq 0 \\
            e^{x}(x + 2) & \text{ if } x < 0
        \end{cases}
    \end{align*}
    Then $\displaystyle \lim_{\xi \rightarrow 0^+}\widehat{f}(\xi) = 1 = \lim_{\xi \rightarrow 0^-}\widehat{f}(\xi)$, $\displaystyle \lim_{\xi \rightarrow 0^+}\widehat{f}'(\xi) = 0 = \lim_{\xi \rightarrow 0^-}\widehat{f}'(\xi)$, and $\displaystyle \lim_{\xi \rightarrow 0^+}\widehat{f}''(\xi) = -1 = \lim_{\xi \rightarrow 0^-}\widehat{f}''(\xi)$, but $\displaystyle \lim_{\xi \rightarrow 0^+}\widehat{f}'''(\xi) = -2 \neq 2 = \lim_{\xi \rightarrow 0^-}\widehat{f}'''(\xi)$.  So $\widehat{f} \in C^2$, but $\widehat{f} \not\in C^3$.
\end{proof}






%%%%%%%%%%%%%%%%%%%%%%%%%%%%%%%%%%%%%%
\problem{Problem 2}{\begin{enumerate}[(a)]
    \item Prove that if $f, g \in \mathcal{S}(\Rl^n)$ (the Schwartz class of functions) then $f * g \in \mathcal{S}(\Rl^n)$.
    \item Find explicitly $\Psi = \widehat{\abs{x}^2} \in \mathcal{S}'(\Rl^n)$.
\end{enumerate}}
\begin{enumerate}[(a)]
    \item \begin{proof}
        First note that the Fourier transform is an isomorphism from $\mathcal{S}(\Rl^n)$ onto itself.  Thus it suffices to show that for $f,g \in \mathcal{S}(\Rl^n)$, $\widehat{f*g}\in \mathcal{S}(\Rl^n)$.  However, $\widehat{f*g} = \widehat{f}\widehat{g} \in \mathcal{S}(\Rl^n)$ since $\widehat{f}$ and $\widehat{g}$ are Schwartz functions and the product of Schwartz functions is a Schwartz function.  Thus $\widehat{f*g} \in \mathcal{S}(\Rl^n)$, which shows $f*g \in \mathcal{S}(\Rl^n)$.
    \end{proof}
        \item \begin{proof}
        something
    \end{proof}
\end{enumerate}






%%%%%%%%%%%%%%%%%%%%%%%%%%%%%%%%%%%%%%
\problem{Problem 3}{Let $0 < \alpha < \frac{n}{2}$.  \begin{enumerate}[(a)]
    \item Prove that $\abs{x}^{-n + \alpha}$ defines a tempered distribution.
    \item Prove that $$\widehat{\abs{x}^{-n + \alpha}}(\xi) = c_{n,\alpha}\abs{\xi}^{-\alpha}.$$  Observe that $\abs{x}^{-n+\alpha}\mathcal{X}_{\{\abs{x} \leq 1\}} \in L^1(\Rl)$ and $\abs{x}^{-n + \alpha}\mathcal{X}_{\{\abs{x} > 1\}} \in L^2(\Rl)$.  Thus $\widehat{\abs{x}^{-n + \alpha}}(\xi)$ is a function.  Show that $\widehat{\abs{x}^{-n + \alpha}}(\xi)$ is radial and homogeneous of order $-\alpha$.
\end{enumerate}
Define the \emph{Hilbert transform} $\Hilb(\phi)$ of a function $\phi \in \mathcal{S}(\Rl)$ by $$ \Hilb(\phi) = \frac{1}{\pi}\pval{\frac{1}{x}} * \phi,$$ where $$\pval{\frac{1}{x}}(\phi) = \lim_{\E \rightarrow 0}\int_{\E < \abs{x} < \frac{1}{\E}}\frac{\phi(x)}{x}\dd x.$$}
\begin{enumerate}[(a)]
    \item\begin{proof}
        \begin{align*}
            \int_\Rl \abs{x}^{-n + \alpha}
        \end{align*}
    \end{proof}
    \item\begin{proof}
    \end{proof}
\end{enumerate}





%%%%%%%%%%%%%%%%%%%%%%%%%%%%%%%%%%%%%%
\problem{Problem 4}{If $\phi \in \mathcal{S}(\Rl)$, prove that $\Hilb(\phi) \in L^1(\Rl)$ if and only if $\widehat{\phi}(0) = 0$.}
\begin{proof}
\end{proof}






%%%%%%%%%%%%%%%%%%%%%%%%%%%%%%%%%%%%%%
\problem{Problem 5}{Prove the following identities:  \begin{enumerate}[(a)]
    \item $\Hilb(fg) = \Hilb(f)g + f\Hilb(g) + \Hilb\qty(\Hilb(f)\Hilb(g))$.
    \item $\Hilb\qty(\mathcal{X}_{(-1,1)}) = \frac{1}{\pi}\log\abs{\frac{x+1}{x-1}}$.
\end{enumerate}}
\begin{proof}
    \begin{enumerate}[(a)]
        \item First note that since
        \begin{align*}
            \widehat{\pval{\frac{1}{x}}} = -i\pi \sgn(\xi),
        \end{align*}
        then the Fourier transform of the Hilbert transform is
        \begin{align*}
            \widehat{\Hilb(\phi)} = \frac{1}{\pi}\widehat{\pval{\frac{1}{x}}}\widehat{\phi} = -i\sgn(\xi)\widehat{\phi}.
        \end{align*}
        Also note that
        \begin{align*}
            \sgn(x - y)\sgn(y) = \sgn(x)\sgn(y) + \sgn(x - y)\sgn(x) - 1
        \end{align*}
        Finally,
        \begin{align*}
            \widehat{\Hilb(f)g + f\Hilb(g) + \Hilb\qty(\Hilb(f)\Hilb(g))} &= \qty[-i\sgn\widehat{f}]*\widehat{g} + \qty[-i\sgn\widehat{g}]*\widehat{f} - i\sgn\qty[\widehat{\Hilb(f)\Hilb(g)}] \\
            &= \qty[-i\sgn\widehat{f}]*\widehat{g} + \qty[-i\sgn\widehat{g}]*\widehat{f} - i\sgn\qty[\qty(-i\sgn\widehat{f})*\qty(-i\sgn\widehat{g})] \\
            &= \int_\Rl -i\sgn(\xi-y)\widehat{f}(\xi - y)\widehat{g}(y)\dd y + \int_\Rl -i\sgn(y)\widehat{g}(y)\widehat{f}(\xi - y) \dd y \\
            &\qquad - i\sgn(\xi)\int_\Rl -\sgn(\xi-y)\widehat{f}(\xi-y)\sgn(y)\widehat{g}(y)\dd y \\
            &= \int_\Rl -i\sgn(\xi-y)\widehat{f}(\xi - y)\widehat{g}(y)\dd y + \int_\Rl -i\sgn(y)\widehat{g}(y)\widehat{f}(\xi - y) \dd y \\
            &\qquad - i\sgn(\xi)\int_\Rl \widehat{f}(\xi - y)\widehat{g}(y)\dd y \\
            &\qquad + i\sgn(\xi)\int_\Rl\sgn(\xi)\sgn(y)\widehat{f}(\xi - y)\widehat{g}(y)\dd y \\
            &\qquad +i\sgn(\xi) \int_\Rl \sgn(\xi - y)\sgn(\xi)\widehat{f}(\xi - y)\widehat{g}(y) \dd y \\
            &= \underline{\int_\Rl -i\sgn(\xi-y)\widehat{f}(\xi - y)\widehat{g}(y)\dd y} \underline{\underline{+ \int_\Rl -i\sgn(y)\widehat{g}(y)\widehat{f}(\xi - y) \dd y}} \\
            &\qquad - i\int_\Rl \sgn(\xi)\widehat{f}(\xi - y)\widehat{g}(y)\dd y \\
            &\qquad \underline{\underline{+ i \int_\Rl \sgn(y)\widehat{f}(\xi - y)\widehat{g}(y)\dd y}} \\
            &\qquad \underline{+i \int_\Rl \sgn(\xi - y)\widehat{f}(\xi - y)\widehat{g}(y) \dd y} \\
            &= -i\sgn(\xi)\int_\Rl \widehat{f}(\xi - y)\widehat{g}(y) \dd y \\
            &= -i\sgn(\xi)\widehat{f}*\widehat{g} = -i\sgn(\xi)\widehat{fg} = \widehat{\Hilb(fg)}
        \end{align*}
        Since the Fourier transform is an isomorphism, the identity holds since we can take the inverse Fourier transform of both sides.
        \item \begin{align*}
            \Hilb\qty(\mathcal{X}_{(-1,1)})(x) &= \frac{1}{\pi}\lim_{\E \rightarrow 0}\int_{\E < y < \frac{1}{\E}} \frac{\mathcal{X}_{-1, 1)}(x - y)}{y} \dd y \\
            &= \frac{1}{\pi}\lim_{\E \rightarrow 0}\int_{\E < \abs{x - y} < \frac{1}{\E}}\frac{\mathcal{X}_{(-1,1)}(y)}{x - y} \dd y \\
            &= \begin{cases}
                \displaystyle\frac{1}{\pi}\lim_{\E\rightarrow 0}\qty[\int_{-1}^{x - \E}\frac{1}{x - y}\dd y + \int_{x + \E}^1 \frac{1}{x - y}\dd y] & \text{ if } x \in (-1, 1) \\[.3cm]
                \displaystyle\frac{1}{\pi}\lim_{\E\rightarrow 0}\int_{-1 + \E}^1 \frac{1}{x - y} \dd y & \text{ if } x = -1 \\[.3cm]
                \displaystyle\frac{1}{\pi}\lim_{\E\rightarrow 0}\int_{-1}^{1 - \E} \frac{1}{x - y} \dd y & \text{ if } x = 1 \\[.3cm]
                \displaystyle\frac{1}{\pi}\lim_{\E\rightarrow 0}\int_{-1}^1 \frac{1}{x - y} \dd y & \text{ if } x \not\in [-1, 1]
            \end{cases} \\
            &= \begin{cases}
                \displaystyle\frac{1}{\pi}\lim_{\E\rightarrow 0}\qty[\underline{-\log\abs{\E}} + \log\abs{x + 1} - \log\abs{x - 1} \underline{+ \log\abs{\E}}] & \text{ if } x \in (-1, 1) \\[.3cm]
                \displaystyle\frac{1}{\pi}\lim_{\E\rightarrow 0} \qty[-\log\abs{x - 1} + \log\abs{x + 1 - \E}] & \text{ if } x = -1 \\[.3cm]
                \displaystyle\frac{1}{\pi}\lim_{\E\rightarrow 0} \qty[-\log\abs{x - 1 + \E} + \log\abs{x + 1}] & \text{ if } x = 1 \\[.3cm]
                \displaystyle\frac{1}{\pi}\lim_{\E\rightarrow 0} \qty[-\log\abs{x - 1} + \log\abs{x + 1}] & \text{ if } x \not\in [-1, 1]
            \end{cases} \\
            &= \begin{cases}
                \displaystyle\frac{1}{\pi} \lim_{\E\rightarrow 0}\log\abs{\frac{x + 1}{x - 1}} & \text{ if } x \in (-1, 1) \\[0.3cm]
                \displaystyle\frac{1}{\pi} \lim_{\E\rightarrow 0}\log\abs{\frac{x + 1 - \E}{x - 1}} & \text{ if } x = -1 \\[0.3cm]
                \displaystyle\frac{1}{\pi} \lim_{\E\rightarrow 0}\log\abs{\frac{x + 1}{x - 1 + \E}} & \text{ if } x = 1 \\[0.3cm]
                \displaystyle\frac{1}{\pi} \lim_{\E\rightarrow 0}\log\abs{\frac{x + 1}{x - 1}} & \text{ if } x \not\in [-1, 1]
            \end{cases} \\
            &= \frac{1}{\pi}\log\abs{\frac{x + 1}{x - 1}} \qquad \forall x \in \Rl
            \end{align*}
        %     \begin{align*}
        %     &= \frac{1}{\pi}\lim_{\E \rightarrow 0}\int_{\E < \abs{y} < 1}\frac{1}{x - y} \dd y \\
        %     &= \frac{1}{\pi}\qty[\lim_{\E \rightarrow 0}\qty[\int_\E^1 \frac{1}{x - y}\dd y + \int_{-1}^{-\E} \frac{1}{x - y}\dd y]] \\
        %     &= \frac{1}{\pi}\qty[\lim_{\E \rightarrow 0} \qty[\log{\abs{x - y}}\Big|_{\E}^1 + \log{\abs{x - y}}\Big|_{-1}^{-\E}]] \\
        %     &= \frac{1}{\pi}\qty[\lim_{\E\rightarrow 0}\qty[\log{\abs{x - 1}} - \log{\abs{x - \E}} + \log{\abs{x + \E}} - \log{\abs{x + 1}}]] \\
        %     &= \frac{1}{\pi}\log{\abs{\frac{x - 1}{x + 1}}}
        % \end{align*}
    \end{enumerate}
\end{proof}






\end{document}
